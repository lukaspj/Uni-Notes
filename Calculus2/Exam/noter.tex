\documentclass{article}
\usepackage{amssymb, amsmath}
\usepackage[utf8]{inputenc}

\title{Calculus 2 noter eksamen}

\begin{document}
\section{Gradient og retningsafledede}
{\tiny Baseret på opgave 2}

$$f(x,y)=13x^2 y^3$$

\subsection{Gradienten}

\subsubsection{Partielle afledede}
Den partielle afledede af funktionen f er:
$$f_x(x,y)=2*13xy^3 = 26xy^3$$

\subsubsection{Beregning af gradienten}
Beregn gradienten af funktionen f.
$$ \triangledown f(x,y)=(f_x(x,y),f_y(x,y))$$
$$ \triangledown f(x,y)=(f_x(x,y),39x^2y^2)$$

\subsubsection{Beregning af gradient i punkt}
Gradienten af f i punktet (1,-1)
$$\triangledown f(1,-1)=(26y^3,39x^2)$$

\subsection{Retningsafledede}

\subsubsection{Enhedsvektoren}
Enhedsvektoren {\bf u} i retningen givet ved vektoren (6,6)
$$u=\frac{1}{\sqrt{6^2 + 6^2}}(6,6)=\frac{1}{\sqrt{72}}(6,6)$$
Enhedsvektoren {\bf u} i retningen givet ved vinklen $\frac{5\pi}{4}$
$$u=(cos(\frac{5\pi}{4}), sin(\frac{5\pi}{4}))=(-\frac{\sqrt{2}}{2}, 
-\frac{\sqrt{2}}{2}) = $$
$$-\frac{\sqrt{2}}{\sqrt{2}*\sqrt{2}}(1,1)=-\frac{1}{\sqrt{2}}(1,1)$$

\subsubsection{Retningsafledede i punkt}
Den retningsafledede $D_u f(1,-1)$ hvor u er enhedsvektoren.
$$D_u f(x,y)= \triangledown f(x,y) * u$$
$$D_u f(1,-1) = 26y^3*\frac{6}{\sqrt{72}} + 39y^2*\frac{6}{\sqrt{72}} = 
\frac{26y^3*6+39x^2*6}{\sqrt{72}}$$
$$\frac{26y^3*6+39x^2*6}{\sqrt{2}*\sqrt{36}} = 
\frac{26y^3*6+39x^2*6}{\sqrt{2}*\sqrt{6}}$$
$$\frac{26y^3+39x^2}{\sqrt{2}} = \frac{26 * (-1)^3+39*1^2}{\sqrt{2}} = 
\frac{13}{\sqrt{2}}$$

\subsubsection{Største og mindste retningsafledede i punkt}
Den enhedsvektor v, der giver den største retningsafledede (den der vokser 
hurtigst) er:
$$u=\frac{\triangledown f(x,y)}{|\triangledown f(x),y|})$$
Den retning hvor f aftager hurtigst er:
$$u=-\frac{\triangledown f(x,y)}{|\triangledown f(x,y)|}$$
\vspace{3mm}
Den største retningsafledede i punktet (1,-1) er:
$$v=\frac{(26y^3,39x^2)}{\sqrt{(26y^3)^2+(39x^2)^2}}$$
$$v=\frac{(-26,39)}{\sqrt{(-26)^2+(39)^2}}$$
$$v=\left(\frac{-26}{\sqrt{(-26)^2+(39)^2}},\frac{39}{\sqrt{(-26)^2+(39)^2}}\right)
 = \frac{1}{\sqrt{13}}(-2,3)$$

\subsubsection{Værdien af den største retningsafledede i punkt}
Værdien af den største retningsafledede $D_uf(x)$ er $|\triangledown f(x)|$\\\\
Værdien af den største retningsafledede af $f$ i punktet (1,-1)
$$D_v f(1,-1)=\sqrt{(26y^3)^2+(39x^2)^2}$$
$$D_v f(1,-1)=\sqrt{(-26)^2+39^2)}=13*\sqrt{13}$$
\newpage
\section{Egenværdier og egenvektorer}
{\tiny Baseret på opgave 2}
$$A=\begin{pmatrix}
1 & 0 & 1\\
1 & 0 & 1\\
1 & 0 & 1
\end{pmatrix}$$

\subsection{Egenværdier}

\subsubsection{Det karakteristiske polynomium}
Det karakteristiske polynomium $P(\lambda)$ for $A$ er:
$$|A-\lambda I| = \begin{vmatrix}
1-\lambda & 0 & 1\\
1 & 0-\lambda & 1\\
1 & 0 & 1-\lambda
\end{vmatrix} = ((1-\lambda)*(-\lambda)*(1-\lambda))-(-\lambda)$$
$$|A-\lambda I| = -\lambda^3+2\lambda^2$$

\subsubsection{Bestemmelse af egenværdierne}
Egenværdierne for A er løsningerne til det karakteristiske polynomium:\\
\begin{displaymath}
solve(P(\lambda),\lambda)
\end{displaymath}
For A bliver det:
$$\lambda_1 = 0, \hspace{5mm} \lambda_2 = 0, \hspace{5mm} \lambda_3 = 2$$

\subsection{Egenvektorer}
Egenvektorer findes ved $(A-\lambda I)u=0$
For vores matrix A med egenværdien $\lambda_2$ bliver det:
$$\begin{pmatrix}
1-0 & 0 & 1\\
1 & 0-0 & 1\\
1 & 0 & 1-0
\end{pmatrix}u=0$$

Hvis vi bringer matricen på echelon form får vi:
$$\begin{pmatrix}x&y&z\end{pmatrix}=\begin{pmatrix}
1 & 0 & 1\\
0 & 0 & 0\\
0 & 0 & 0
\end{pmatrix}u=0$$
Vi kan se at der kun er 1 pivot, altså den første kolonne og der er 2 frie 
variable. Derfor får vi 2 egenvektorer.\\
Den første egenvektor finder vi ved at tage den første frie variable, der er 
den anden kolonne (eller y-variablerne) og flytter y over på den anden side af 
lighedstegnet.\\
Hvis vi gør det for alle den første og den anden række og sætter $u_y=1$ får vi:
$$u_x=x+z=0$$
$$u_y=1$$
$$u_z=0=0$$
Og vi får således den første egenvektor til at være:
$$\begin{pmatrix}0\\1\\0\end{pmatrix}$$
Det samme gør vi så for den tredje kolonne altså z.
$$v_x=x=-z$$
$$v_y=0=0$$
$$v_z=1$$
$$\begin{pmatrix}-1\\0\\1\end{pmatrix}$$
Det er så de to egenvektorer.

\subsubsection{Egenrum}
Egenrummet udspændes af alle vektorer af formen $t*u$ hvor u er egenvektoren.
For vores matrix bliver det $t(0,1,0)$ og $t(-1,0,1)$
Og vi får så vores egenrum til at blive:
$$span(
\begin{pmatrix}0\\1\\0\end{pmatrix},
\begin{pmatrix}-1\\0\\1\end{pmatrix}
)$$

\subsubsection{Beregning af vektor når u er en egenvektor for matricen A}
$u$ er en egenvektor for matricen A hørende til egenværdien $\lambda_1=0$. Så 
kan vi beregne følgende vektor:
$$100A^2u+50u=?u$$
$$Au=\lambda u$$
$$100\lambda^2u+50=?u$$
$$100*0^2u+50u=?u$$
$$50u=50u$$

\newpage

\section{Dobbeltintegraler}
{\tiny Baseret på opgave 2}
\subsection{Partiel integration}
For at beregne værdien af det partielle integral
$$\int\limits_{min}^{max}f(x)dx$$
Integrerer vi udtrykket og trækker den nedre grænse fra den øvre grænse.
$$\int\limits_{min}^{max}f(x)dx = [F(x)]^{max}_{min} = F(max) - F(min)$$

$$\int\limits_{0}^{\sqrt[4]{16-x^4}}4y^3dy = [x^4]^{\sqrt[4]{16-x^4}}_0 = 
\sqrt[4]{16-x^4}^4 - 0^4=16-x^4$$

\newpage

\section{Overbestemte ligningssystemer}
{\tiny Baseret på opgave 2}\\
\begin{center}
\begin{tabular}{c | c | c | c | c}
x & -2 & 0 & 2 & 4\\
\hline
y & 1 & 2 & 4 & 5
\end{tabular}
\end{center}
\subsection{Opstilling af overbestemt ligningssystem}
For at opstille et ligningssystem på formen Ac=y til bestemmelse af vektoren 
$c=(c_0,c_1)^T$ laver vi følgende matricer:
$$\begin{pmatrix}
1&x_1\\
...\\
1&x_n
\end{pmatrix}
\begin{pmatrix}
c_0\\c_1
\end{pmatrix}
=
\begin{pmatrix}
y_1\\
...\\
y_n
\end{pmatrix}
$$
For den givne tabel bliver det:
$$\begin{pmatrix}
1&-2\\
1&0\\
1&2\\
1&4
\end{pmatrix}
\begin{pmatrix}
c_0\\c_1
\end{pmatrix}
=
\begin{pmatrix}
1\\
2\\
4\\
5
\end{pmatrix}
$$

\subsection{Normalligningerne}
For at udregne normalligningerne $A^TAc=A^Ty$ er det lettest først at finde 
$A^T$ og gange det ind på A og y og til sidst få en ligning på formen:
$$A^TA
\begin{pmatrix}
c_0\\c_1
\end{pmatrix} = A^Ty$$
$A^T$ for den givne tabel er:
$$\begin{pmatrix}
1 & 1 & 1 & 1\\
-2 & 0 & 2 & 4
\end{pmatrix}$$
Så kan vi finde $A^TA og A^Ty$
$$A^TA=\begin{pmatrix}
4 & 4\\
4 & 24
\end{pmatrix}$$
$$A^Ty=\begin{pmatrix}
12\\
26
\end{pmatrix}$$
Vi får så normalligningen til:
$$\begin{pmatrix}
4 & 4\\
4 & 24
\end{pmatrix}
\begin{pmatrix}
c_0\\c_1
\end{pmatrix} =
\begin{pmatrix}
12\\26
\end{pmatrix}$$
For at finde $c_0$ og $c_1$ ved hjælp af normalligningerne løser vi de to 
ligninger:
$$4c_0+4c_1=12$$
$$4c_0+24c_1=26$$
$$c_0 = \frac{23}{10}, \hspace{5mm} c_1 = \frac{7}{10}$$

\subsection{Afvigelsesvektoren}
Afvigelsesvektoren findes ved $Ac-y$
For vores tabel bliver det:
$$\begin{pmatrix}
1&-2\\
1&0\\
1&2\\
1&4
\end{pmatrix}
\frac{1}{10}\begin{pmatrix}
23\\7
\end{pmatrix}
-
\begin{pmatrix}
1\\
2\\
4\\
5
\end{pmatrix}
=
\frac{1}{10}\begin{pmatrix}
-1\\
3\\
-3\\
1
\end{pmatrix}$$

\newpage

\section{Second derivatives test}
{\tiny Baseret på opgave 2}
$$f(x,y)=1+3x-2y-x^3+y^2$$

\subsection{Kritiske punkter}
For at finde de kritiske punkter finder vi de partielle afledte og sætter dem 
begge lig med 0.
$$f_x = 0$$
$$f_y = 0$$
Og finder så alle løsninger til ligningssystemet.\\
For funktionen $f$ bliver det:
$$f_x(x,y) = -3x^2+3$$
$$f_y(x,y) = 2y-2$$
$$-3x^2+3=0$$
$$2y-2=0$$
Så får vi:
$$(-1,1), \hspace{5mm} (1,1)$$

\subsection{Den kritiske værdi}
Den kritiske værdi kan findes ved at sætte det kritiske punkt ind i funktionen. 
For funktionen $f$ bliver det:
$$f(-1,1) = 1-3-2+1+1 = -2$$

\subsection{Second derivatives test}
Man kan udregne teststørrelsen D i andenordenskriteriet ved at finde de dobbelt 
afledte og sætte dem ind i følgende ligning:
$$D=f_{xx}(x,y)f_{yy}(x,y)-(f_{xy}(x,y))^2$$
For funktionen $f$ bliver det:
$$f_{xx} = -6x$$
$$f_{yy} = 2$$
$$f_{xy} = 0$$
$$D=-6x*2-0 = -12x$$

\subsection{Arten af det kritiske punkt}
For at finde arten af det kritiske punkt kigger man på differential kvotienten 
og $f_{xx}$.\\
\renewcommand{\arraystretch}{1.4}
\begin{center}
\begin{tabular}{c | c | c}
D & $f_{xx}$ & art\\
\hline 
$D > 0$ & $f_{xx}(x,y) > 0$ & $f(x,y)$ er et lokalt minimum\\
\hline
$D > 0$ & $f_{xx}(x,y) < 0$ & $f(x.y)$ er et lokalt maksimum\\
\hline
$D < 0$ & Irrelevant & $f(x,y)$ er et saddelpunkt\\
\hline
$D = 0$ & Irrelevant & Det er ikke til at sige ud fra denne test

\end{tabular}
\end{center}

\newpage

\section{Taylorrækker}
{\tiny Baseret på opgave 8923}

\subsection{Taylorrækker}
En taylor serie er defineret som:
$$f(x)=\sum\limits_{n=0}^{\infty}\frac{f^{(n)}(a)}{n!}(x-a)^n = f(a) + 
\frac{f'(a)}{1!}x+\frac{f''(a)}{2!}x^2+...$$

\subsection{Maclaurinrækken}
En Maclaurin række er en taylor række hvor man tager $f(0)$\\
En Maclaurinrække er defineret som:
$$f(x)=\sum\limits_{n=0}^{\infty}\frac{f^{(n)}(0)}{n!}x^n = f(0) + 
\frac{f'(0)}{1!}x+\frac{f''(0)}{2!}x^2+...$$
For $cos(x)$ er en Maclaurin række defineret som:
$$\sum\limits_{n=0}^{\infty}(-1)^n\frac{x^{2n}}{(2n)!}=1-\frac{x^2}{2!}+\frac{x^4}{4!}-\frac{x^6}{6!}...$$
For $sin(x)$ er en Maclaurin række defineret som:
$$\sum\limits_{n=0}^{\infty}(-1)^n\frac{x^{2n+1}}{(2n+1)!}=x-\frac{x^3}{3!}+\frac{x^5}{5!}-\frac{x^7}{7!}...$$

\subsection{Potensrækker}
En potensrækker er defineret som:
$$\sum\limits_{n=0}^{\infty}C_n(x-a)^n=C_0+C_1(x-a)+C_2(x-a)^2+...$$

\newpage

\section{Lagrange ligninger}
{\tiny Baseret på opgavesæt 7543}
$$f(x,y)=x^2+y^2-xy, \hspace{5mm} g(x,y)=e^{x+y}$$

\subsection{Opstilling af Lagrange ligningen}
For at opstille en Lagrange ligning med bibetingelsen $g(x,y)=k$ i et punkt 
$(x_0,y_0)$ udnytter man at:
$$\triangledown f(x,y) = \lambda \triangledown g(x,y)$$
og
$$g(x,y)=k$$
Så får man 3 ligninger:
$$f_x=\lambda g_x, \hspace{5mm} f_y=\lambda g_y, \hspace{5mm} g(x,y)=k$$
Hvis vi gør det for funktionen f får vi:
$$\triangledown f=(2x-y, 2y-x), \hspace{5mm} \triangledown g=(e^{x+y},e^{x+y})$$
$$2x-y=\lambda e^{x+y}$$
$$2y-x=\lambda e^{x+y}$$
$$e^{x+y}=k$$
Og det er så vores tre Lagrange ligninger

\subsection{Ekstemumspunkter}
I de tre Lagrange ligninger er der 3 ubekendte, derfor kan man finde 
ekstremumspunkterne ved at finde alle løsninger til de 3 ligninger.
Hvis vi gør dette for f under bibetingelsen $g(x,y)=e^2$ får vi:
$$2x-y=\lambda e^{x+y}$$
$$2y-x=\lambda e^{x+y}$$
$$e^{x+y}=e^2$$
Når vi løser dette ligningssystem får vi punktet $(1,1)$ som er et muligt 
ekstremumspunkt\\[5mm]
Vi kan se at det er et minimumspunkt da funktionen er uendeligt voksende og 
derfor kun vil have et enkelt ekstremumspunkt som er et minimums punkt

\newpage

\section{Ortogonale projektioner}
{\tiny Baseret på opgavesæt 4111}\\
Følgende vektorer er i rummet $\Re^3$
$$u=(1,2,2), \hspace{5mm} x=(1,3,1)$$

\subsection{Længden af vektoren}
Her bruger man blot Pythagoras sætning:
$$||v||=\sqrt{v_x^2+v_y^2+v_z^2}$$
For vektoren x bliver det:
$$||x||=\sqrt{1^2+3^2+1^2}=\sqrt{11}$$

\subsection{Afstand mellem vektorer}
Afstanden mellem to vektorer, givet to vektorer i rummet $\Re^3$ $u$ og $v$ er 
defineret som $||u-v||$:
$$||u-v|| = \sqrt{(u_x-v_x)^2+(u_y-v_y)^2+(u_z-v_z)^2}$$
For de givne vektorer x og u bliver det:
$$||u-x||=\sqrt{(1-1)^2+(2-3)^2+(2-1)^2}$$

\subsection{Skalarprodukt}
Skalarproduktet eller prikproduktet mellem to vektorer, givet to vektorer i 
rummet $\Re^3$ $u$ og $v$ er defineret som:
$$u\cdot v = u_xv_x+u_yv_y+u_zv_z$$
For de givne vektorer x og u bliver det:
$$u\cdot x = 1*1 + 2*3 + 2*1 = 9$$

\subsection{Den ortogonale projektion}
Den ortogonale projektion af p af vektoren $\omega$ på underrummet U kan 
skrives som den linearkombination hvor koefficienterne er løsningen på 
ligningssystemet. Hvis underrummet har dimensionen 3 kan det skrives som:
$$p=\alpha_1v_1+\alpha_2v_2+\alpha_3v_3$$
Hvilket kan skrives som formlen:
$$p=A(A^TA)^-1A^T\omega$$
Hvor $A=(v_1,v_2,v_3)$ og $v_n$ er basisvektorer for underrummet.\\
Den ortogonale projektion $v$ af vektoren $x$ på underrummet $U$ udspændt af 
vektoren $u$ kan da findes ved:
$$u(u^Tu)^{-1}u^tx=\begin{pmatrix}
1\\2\\2
\end{pmatrix}$$
Da $A=U=span(u)=u$

\subsection{Den mindste afstand til et underrum}
Givet projektionen $p$ af vektorn $v$ på underrummet $U$ er den mindste afstand:
$$||v-p||$$
For den givne vektor $x$ med projektionen $p$ på underrummet $U$ bliver det:
$$||x-p||\sqrt{(1-1)^2+(3-2)^2+(1-2)^2}=\sqrt{(2)}$$
\newpage

\section{Differential ligningssystemer}
{\tiny Baseret på Opgavesæte 7543}
$$y_1'=ay_1+y_2$$
$$y_2'=ay_1+y_2$$
$$u_1=\begin{pmatrix}
1\\1
\end{pmatrix}$$
$$u_2=\begin{pmatrix}
1\\-a
\end{pmatrix}$$
$$\lambda_1 = a+1, \hspace{5mm} \lambda_2=1$$
\subsection{Løsning af systemer på formen $x'=A*x$}
Hvis et ligningssystem har form $x'=Ax$ er løsningen:
$$x(t)=Ce^{At}$$
$\lambda_n$ og $v_n$ er henholdsvist systemets egenværdier og egenvektorer. Vi 
kan så finde den fuldstændige løsning ved:
$$x(t)=C_1e^{\lambda_1t}v_1+C_2e^{\lambda_2t}v_2+...$$
Vi kan se at det givne ligningssystem kan skrives som:
$$\begin{pmatrix}
y_1'(t)\\
y_2'(t)
\end{pmatrix}=
\begin{pmatrix}
a & 1\\
a & 1
\end{pmatrix}\begin{pmatrix}
y_1(t)\\
y_2(t)
\end{pmatrix}$$
Hvilket passer på $y'=A*y$\\
Vi kan så finde den fuldstændige løsning til ligningssystemet:
$$y(t)=C_1e^{(1+a)t}u_1+C_2e^{1*t}u_2$$

\subsection{Linearkombinationer}
{\tiny Opgave 4111 7.5}\\
En vektor der kan skrives på formen $x_1v_1+x_2v_2$ med $x_1,x_2\in\Re$ kaldes 
en linearkombination af vektorerne $v_1$ og $v_2$.\\
Mængden af alle linearkombinationerne af $v_1$ og $v_2$ betegnes med 
$span(v_1,v_2)$
Hvis vi har givet i linearkombinationen: $v=C_1u_1+C_2u_2$ har givet vektorerne 
$v, u_1$ og $u_2$ kan vi finde $C_1$ og $C_2$ som to ligninger med to ubekendte

\newpage

\section{Polære koordinater}
{\tiny Baseret på opgave 7543}
$$D=sin(x^2+y^2)$$
\subsection{Området D i polære koordinater}
Når man skal beskrive et område i polærekoordinater bruger man formen:
$$\{(r,\theta)|u\leq\theta\leq v, R_1\leq r\leq R_2 \}$$
Hvor u er den mindste vinkel (i radianer) og v er den største. $R_1$ er den 
mindste radius og $R_2$ er den første.\\
For D bliver det:
$$\{(r,\theta)|0\leq\theta\leq 2\pi, 1\leq r\leq 3 \}$$


\newpage

\section{Matrix regning}
\subsection{Matrix multiplikation ($A*B$)}
For at gange to matricer sammen skal man gange rækker med søjler.
Man tager det første tal i den første række i den første matrix og ganger det 
ind på det første tal i den første kolonne i den anden matrix og lægger det 
sammen med det andet tal i den første række i den første matrix og lægger det 
sammen med det andet tal i den anden kolonne i den anden matrix for at finde 
det andet tal i anden kolonne i første række.\\
Altså:
\begin{verbatim}
1. række * 1. søjle = 1.tal i 1. række.
2. række * 1. søjle = 1.tal i 2. række.
1. række * 2. søjle = 2.tal i 1. række.
3. række * 2. søjle = 2.tal i 3. række.
\end{verbatim}
\subsection{Transponering ($A^T$)}
Når man transponerer en matrix spejlvender og roterer man den. Dvs.
$$\begin{pmatrix}
1 & 2\\
3 & 4\\
5 & 6
\end{pmatrix}^T
\sim
\begin{pmatrix}
2 & 1\\
4 & 3\\
6 & 5
\end{pmatrix}
\sim
\begin{pmatrix}
1&3&5\\
2&4&6
\end{pmatrix}
$$


Note: $A*B=A^TB$
\newpage
\section{Pensum}
\begin{description}
\item [{[L]} 5] Projektioner og ortogonalitet
\item [{[L]} 6] Egenværdier og egenvektorer
\item [{[S]} 8.5] Potensrækker
\item [{[S]} 8.6] Represæntation af funktioner ved potensrækker
\item [{[S]} 8.7] Taylor- og Maclaurin rækker
\item [{[S]} 11.6] Retningsafledet og gradient vektoren
\item [{[S]} 11.7] Maksimums- og minimumsværdier
\item [{[S]} 12.3] Dobbeltintegraler over generelle områder
\item [{[S]} 12.4] Dobbeltintegraler i polære koordinater
\end{description}
\end{document}