\documentclass{article}
\usepackage{amssymb, amsmath}
\usepackage[utf8]{inputenc}
\usepackage{alltt}
\renewcommand{\ttdefault}{txtt}

\title{E4DSA eksamensnoter}

\begin{document}
\maketitle

\section{Analyse af digitale signaler med DFT og spektrogram}
DFT i eksponential form:
\begin{equation}
X(m)=\sum\limits_{n=0}^{N-1}x(n)e^{-j2\pi nm/N}
\end{equation}
I rektangulær form:
\begin{equation}
X(m)=\sum\limits_{n=0}^{N-1}x(n)\cdot \left[\cos(2\pi nm/N) - j\cdot \sin(2\pi nm/N)\right]
\end{equation}
Eulers relation:
\begin{equation}
e^{-j\theta} = \cos(\theta)-i\sin(\theta)
\end{equation}
DFT er baseret på at signalet ikke ændrer sig over tid, og det er upraktisk at køre DFT på store signaler.

Derfor spektrogram:
\begin{equation}
X(k,n)=\sum\limits_{m=0}^{L-1}w(m)x(n+m)e^{-j(2\pi k/N)m}
\end{equation}
Vindue og DFT på forskellige dele af signalet.

\section{Spektral forbredning, zero-pedding og window funktioner i relation til DFT}
DFT er begrænset til:
\begin{equation}
f_{analysis}(m)=\frac{mf_s}{N}, m=0,\dots,N-1
\end{equation}
DFT'en er kun præcis hvis alle frekvenserne i signalet passer ind i $\frac{mf_s}{N}$. Hvis der er en sinus med en frekvens der ligger lidt ved siden af, vil der være lækage.

Zero-padding gør at man kan interpolere i DFT-resultatet men det giver ikke bedre frekvensopløsning. N-zeros giver dobbelt så mange bins i DFT'en. Kan bruges til at visualisere vinduers lækage, da $\frac{mf_s}{N*x}$

Vinduer reducerer de sidelobes der kommer af lækage, det rektangulære vindues pludselige skift fra 1 til 0 er grunden til lækagen, et bedre vindue kan reducere lækage.

\section{FIR/IIR filter analyse og design vha. pladsering af poler/nuller}
Normaliseret frekvens er vinkel af frekvensen:
\begin{equation}
\theta = 2\cdot\pi\cdot\frac{f_c}{f_s}
\end{equation}
Så kan man finde filter koefficienterne ved:
$$H(z)=\frac{G\cdot (z-z_0)\cdot(z-z_1)}{(z-p_0)\cdot(z-p_1)}$$
Lavpas: Begge nuller i $-1$, højpas: begge nuller i $1$, bandreject: nuller på periferien med samme vinkel som polerne, højpas: nuller i 1 og -1.


\section{Windowmethod til FIR filter design}
Design det optimale filter i frekvens-domænet, kør en invers-DFT så får man impulsresponsen for det optimale filter. Dette impulsrespons er dog uendeligt så man skal beslutte sig for en vindues-længde, der bliver én mindre end antallet af koefficienter i filteret.

Herefter skal man påføre et vindue for at forbedre impulsresponsen, da det rektangulære vindue har ripples.
\\
\\
Et all-pass filter er blot det filter hvor center frekvensen ($b_0$) er 1 og resten 0.
\\
Et high-pass filter designet ved at trække et low-pass filter fra et all-pass filter.
\\
Et band-pass filter kan designes ved at shifte et lav-pas filter (gange med sinus af $f_c$)

\section{Bilinear transformation IIR filter design}
2nd-order chebychev lowpass filter:
$$H_c(s)=\frac{c}{s^2+bs+c}$$
Hvor: $b=137.945$, $c=17410.145$.\\
Så indsætter man: $\frac{2}{t_s}\cdot\left(\frac{1-z^{-1}}{1+z^{-1}}\right)$ i stedet for s, for at mappe fra s- til z-plan:
$$H(z)=\frac{c}{\left(\frac{2}{t_s}\cdot\left(\frac{1-z^{-1}}{1+z^{-1}}\right)\right)^2 + b\left(\frac{2}{t_s}\cdot\left(\frac{1-z^{-1}}{1+z^{-1}}\right)\right)^2+c}$$


\section{Interpolation og decimation}
Decimation:
$$Lavpas\rightarrow Decimer$$
Interpolation:\\
Udfyld med $L-1$ 0'er ind imellem samples. Lavpas filtrer for at fylde 0'erne.\\
Up-sampling med ikke hel-tal:
\begin{itemize}
\item Udregn således at $faktor = \frac{L}{M}, L>M$
\item Up-sample med faktor L
\item Down-sample med faktor M
\end{itemize}
Down-sampling med ikke hel-tal:
\begin{itemize}
\item Udregn således at $faktor = \frac{M}{L}, M>L$
\item Up-sample med faktor L
\item Down-sample med faktor M
\end{itemize}

\section{Stokastiske signaler, herunder middelværd, varians, sandsynligheds-tæthedsfunktion og histogram}
Middelværdi:
$$x_{ave}\frac{x(1,\dots,N)}{N}$$
Varians:
$$\sigma^2=\frac{1}{N}\sum_{n=1}^{N}\left[x(n)-x_{ave}\right]^2$$
$\sigma$ er gennemsnittet af afvigelsen fra middelværdien.
\\
Bessel's correction:
$$\sigma^2_{unbiased}=\frac{1}{N-1}\sum_{n=1}^{N}\left[x(n)-x_{ave}\right]^2$$
\\
Histogram er muligheder ud af x-aksen og antallet af optrædender op af y-aksen.
\section{Beregning af SNR i tids- og frekvens-domænet}
\subsection{Tidsdomæne:}
$$SNR=\frac{Signal effekt}{Noise effekt}$$
Da signal og støj er uafhængigt så er: 
$$x(n)=x_n(n)+x_s(n)$$
Og da har man at:
$$SNR=\frac{\sum\limits^{N-1}_{n=0}\left[x_s(n)\right]^2}{\sum\limits^{N-1}_{n=0}\left[x_n(n)\right]^2}$$
Og vi har også at:
$$\sigma^2=\sigma^2_n + \sigma^2_s$$
Hvis signalet har 0 ved middelværdi f.eks. sinuskurve der er symmetrisk omkring x-aksen.
$$SNR=\frac{\sigma_s^2}{\sigma_n^2}$$
\subsection{Frekvensdomæne:}
Estimat:
$$SNR=\frac{\sum |X(m)|^2 \text{ iff. } |X(m)| > Threshold}{\sum |X(m)|^2 \text{ iff. } |X(m)| < Threshold}$$
Threshold er en vurderingssag.
\section{Midlingsfiltre}
Bruges til at fjerne støj,\\
FIR:
$$y(n)=\frac{1}{N}\left[x(n),\dots,x(n-N+1)\right]$$
IIR:
$$y(n)=\frac{1}{N}\left[x(n)-x(n-N)\right]+y(n-1)$$
$$H_{ma}(z)=H_{rma}(z)$$
Output noise power:
$$\sigma^2_{out}=\frac{\sigma^2_{in}}{N}$$\\
\\
Exponentielt midlingsfilter har hurtigere responstid og kan have et N der ikke er et heltal.
$$y(n)=\alpha x(n)+ (1-\alpha)y(n-1)$$
$$\alpha=\frac{2}{N-1}$$
Jo højere $\alpha$ jo skarpere indsvingningstid men mere støj. Skift alpha over tid er tit brugt i praksis.
\section{Case 1 - FSK transmission}
\section{Case 2 - Audio filter}
\section{Case 3 - Vejecelle}
\end{document}