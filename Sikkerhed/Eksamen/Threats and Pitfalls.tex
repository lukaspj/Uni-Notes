% Dokumentklassen s�ttes til memoir.
% Manual: http://ctan.org/tex-archive/macros/latex/contrib/memoir/memman.pdf
\documentclass[a4paper,oneside,article]{memoir}

\usepackage{pgf}
\usepackage{tikz}
\usepackage{pgfplots}
\usetikzlibrary{arrows,automata, positioning}
\usepackage{verbatim}
 
% Danske udtryk (fx figur og tabel) samt dansk orddeling og fonte med
% danske tegn. Hvis LaTeX brokker sig over æ, ø og å skal du udskifte
% "utf8" med "latin1" eller "applemac". 
\usepackage[latin1]{inputenc}
\usepackage[danish]{babel}
\usepackage[T1]{fontenc}
 
% Matematisk udtryk, fede symboler, theoremer og fancy ting (fx k�debr�ker)
\usepackage{amsmath,amssymb}
\usepackage{bm}
\usepackage{amsthm}
%\usepackage{mathtools}
 
% Kodelisting. Husk at l�se manualen hvis du vil lave fancy ting.
% Manual: http://mirror.ctan.org/macros/latex/contrib/listings/listings.pdf
\usepackage{listings}
 
% Fancy ting med enheder og datatabeller. L�s manualen til pakken
% Manual: http://www.ctan.org/tex-archive/macros/latex/contrib/siunitx/siunitx.pdf
%\usepackage{siunitx}

% Inds�ttelse af grafik.
\usepackage{graphicx}
\usepackage{float}
\usepackage{caption}
 
% Reaktionsskemaer. L�s manualen for at se eksempler.
% Manual: http://www.ctan.org/tex-archive/macros/latex/contrib/mhchem/mhchem.pdf
%\usepackage[version=3]{mhchem}
%\usepackage[noend]{algpseudocode}
%\usepackage{algorithm}

\usepackage{xcolor,colortbl}

\usepackage{listings}

\definecolor{javared}{rgb}{0.6,0,0} % for strings
\definecolor{javagreen}{rgb}{0.25,0.5,0.35} % comments
\definecolor{javapurple}{rgb}{0.5,0,0.35} % keywords
\definecolor{javadocblue}{rgb}{0.25,0.35,0.75} % javadoc

\lstset{language=Java,
basicstyle=\small, %\ttfamily,
keywordstyle=\color{javapurple}\bfseries,
stringstyle=\color{javared},
commentstyle=\color{javagreen},
morecomment=[s][\color{javadocblue}]{/**}{*/},
numbers=left,
numberstyle=\tiny\color{black},
stepnumber=1,
numbersep=10pt,
tabsize=4,
showspaces=false,
showstringspaces=false}

\newcommand{\notimplies}{%
  \mathrel{{\ooalign{\hidewidth$\not\phantom{=}$\hidewidth\cr$\implies$}}}}

\begin{document}
    \title{dSik - Threats and Pitfalls}
    \author{Lukas Peter J�rgensen, 201206057, DA4
            }
    \maketitle
    
    \chapter{Sikkerhedsm�l}
    \section*{CAA}
    CAA st�r for:
    \begin{description}
    \item[Confidentiality:]Information skal holdes 
    hemmelig for uvedkommende, g�lder b�de under 
    forsendelse, opbevaring og behandling af data
    \item[Authenticity:]Informationen er autentisk, 
    den er ikke blevet manipuleret af en uautoriseret 
    person.
    \item[Availability:]Systemer skal v�re tilg�ngelig 
    n�r de skal bruges
    \end{description}
    
    \section*{Definition af et sikkert system}
    Det er typisk sv�rt eller umuligt at bevise et 
    system er sikkert. Tit bliver man n�dt til at
    lave mange antagelser om angriberens muligheder 
    for at bevise at et system er sikkert og disse
    antagelser er typisk forkerte.
    
    Vi bruger udtrykket "sikret system" istedet for
    "sikkert system" da det blot indikerer at vi har
    sikret systemet mod nogle bestemte tilf�lde. Disse
    tilf�lde defineres ved en $Sikkerhedspolitik$ p�
    baggrund af en $Trusselmodel$, og implementerer 
    herefter denne sikkerhedspolitik vha. nogle 
    $Sikkerhedsmekanismer$. P� den m�de f�r vi, at et 
    sikret system kan beskrives som:
    $$Sikret system = Sikkerhedspolitik + Trusselsmodel 
    + Sikkerhedsmekanismer$$
    
    \chapter{Klassificering af angreb}
    \section*{STRIDE}
    Kategorisering ud fra en angribers m�l.
    \begin{description}
    \item[S]poofing Identity: At angriberen f�r mulighed
    for at udgive sig for at v�re en anden.
    \item[T]ampering: At angriberen f�r mulighed for at
    manipulere data uden at dette bliver opdaget.
    \item[R]epudiation: At angriberen f�r mulighed for
    at kunne n�gte at have gjort noget h�n har gjort.
    \item[I]nformation Disclosure: At angriberen f�r 
    adgang til data han/hun ikke burde se.
    \item[D]enial of Service: 
    \item[E]levation of privilege:
    \end{description}
    Nogle gange kan angriberen godt have et m�l,
    men alligevel f�r angrebet andre konsekvenser
    end m�let.
    
    \section*{X.800}
    Meget begr�nset og ufyldestg�rende opdeling.
    Opdeling i hvordan angrebet bliver opn�et:\\
    \textbf{Passive angreb:}
    \begin{description}
    \item[Eavesdropping:]
    \item[Traffic Analysis:]
    \end{description}
    \textbf{Aktive angreb:}
    \begin{description}
    \item[Replay:]
    \item[Modification:]
    \end{description}
    \section*{EINOO}
    Opdeling i hvorfra og hvem der angriber, EI er
    "Hvem" og NOO er "Hvorfra"
    \begin{description}
    \item[E]xternal attackers:
    \item[I]nsiders:
    \end{description}
    \noindent\makebox[\linewidth]{\rule{\paperwidth}{0.4pt}}
    \begin{description}
    \item[N]etwork attacks: Lytte og modificere 
    netv�rkstraffik.
    \item[O]ffline attacks: Uautoriseret adgang til
    der er permanent lagret i systemet.
    \item[O]nline attacks: Angriberen bryder ing og
    overv�ger systemet mens det k�rer, som f.eks. at
    afl�se hemmelig information fra RAM'en
    \end{description}
    
    \section*{TPM}
    Opdeling ud fra hvilket forsvar der har fejlet:
    
    \begin{description}
    \item[T]hreat Model: Angrebet var muligt fordi
    vores trusselsmodel var ukomplet.
    \item[P]olicy: Angrebet var muligt fordi vores 
    sikkerhedspolitik kommunikerede noget andet end
    vi �nskede.
    \item[M]echanism: Angrebet var muligt fordi
    angriberen kunne bryde uden om vore 
    sikkerhedsmekanismer.
    \end{description}
    
    \chapter{Buffer overflows}
    D�rlig programmering kan f�re til at man kan
    overflowe en buffer og derved �ndrer i noget det
    ikke var meningen man skulle �ndre i.
    
    \chapter{Cross-Site Scripting}
    Injecte et script p� en anden side gennem et
    bruger input. F.eks. for at stj�le cookie.
    
    \chapter{SQL Injections}
    Injecte noget SQL igennem input felterne som f.eks.
    brugernavnet.
    
    \chapter{Covert Channels}
    At kommunikere gennem kanaler man ikke burde kunne 
    kommunikere igennem.
    
\end{document}