% Dokumentklassen sættes til memoir.
% Manual: http://ctan.org/tex-archive/macros/latex/contrib/memoir/memman.pdf
\documentclass[a4paper,oneside,article]{memoir}

\usepackage{pgf}
\usepackage{tikz}
\usepackage{pgfplots}
\usetikzlibrary{arrows,automata}
\usepackage{verbatim}
 
% Danske udtryk (fx figur og tabel) samt dansk orddeling og fonte med
% danske tegn. Hvis LaTeX brokker sig over æ, ø og å skal du udskifte
% "utf8" med "latin1" eller "applemac". 
\usepackage[utf8]{inputenc}
\usepackage[danish]{babel}
\usepackage[T1]{fontenc}
 
% Matematisk udtryk, fede symboler, theoremer og fancy ting (fx kædebrøker)
\usepackage{amsmath,amssymb}
\usepackage{bm}
\usepackage{amsthm}
%\usepackage{mathtools}
 
% Kodelisting. Husk at læse manualen hvis du vil lave fancy ting.
% Manual: http://mirror.ctan.org/macros/latex/contrib/listings/listings.pdf
\usepackage{listings}
 
% Fancy ting med enheder og datatabeller. Læs manualen til pakken
% Manual: http://www.ctan.org/tex-archive/macros/latex/contrib/siunitx/siunitx.pdf
%\usepackage{siunitx}

% Indsættelse af grafik.
\usepackage{graphicx}
\usepackage{float}
\usepackage{caption}
%\usepackage{subcaption}
 
% Reaktionsskemaer. Læs manualen for at se eksempler.
% Manual: http://www.ctan.org/tex-archive/macros/latex/contrib/mhchem/mhchem.pdf
%\usepackage[version=3]{mhchem}
%\usepackage[noend]{algpseudocode}
%\usepackage{algorithm}

\usepackage{xcolor,colortbl}

\usepackage{listings}

\definecolor{javared}{rgb}{0.6,0,0} % for strings
\definecolor{javagreen}{rgb}{0.25,0.5,0.35} % comments
\definecolor{javapurple}{rgb}{0.5,0,0.35} % keywords
\definecolor{javadocblue}{rgb}{0.25,0.35,0.75} % javadoc

\lstset{language=Java,
basicstyle=\small, %\ttfamily,
keywordstyle=\color{javapurple}\bfseries,
stringstyle=\color{javared},
commentstyle=\color{javagreen},
morecomment=[s][\color{javadocblue}]{/**}{*/},
numbers=left,
numberstyle=\tiny\color{black},
stepnumber=1,
numbersep=10pt,
tabsize=4,
showspaces=false,
showstringspaces=false}

\newcommand{\notimplies}{%
  \mathrel{{\ooalign{\hidewidth$\not\phantom{=}$\hidewidth\cr$\implies$}}}}
  
\newcommand{\inner}[2]{\langle #1,#2 \rangle}

\begin{document}
    \title{Lineær algebra noter - Egenværdier og vektorer}
    \author{Lukas Peter Jørgensen, 201206057, DA4
            }
    \maketitle
    
    \tableofcontents
        
    \chapter{Disposition}
    \begin{enumerate}
    \item TBD
    \end{enumerate}
    
	\chapter{Noter}
	
	\section{Egenværdi og egenvektor}
	$$A\in Mat_{m,n}(\mathbb{F}), \lambda \in \mathbb{F}$$
	
	$\lambda$ er en egenværdi (eller karakteristisk værdi) for matricen 
	$A$ hvis den opfylder:
	$$Ax=\lambda x$$
	Hvor $x$ egenvektoren ikke er nulvektoren $0$ og kaldes egenvektoren 
	tilhørende egenværdien $\lambda$.
	
	En egenværdi har adskillige tilknyttede egenvektorer (man kan f.eks. 
	blot gange en skalar på), men en egenvektor har altid kun én egenværdi. 
	
	Komplekse egenværdier har den egenskab at hvis $\lambda = a+bi$ er en
	egenværdi for $A$, vil den konjungerede også være en egenværdi for $A$.
	
	\section{Egenrummet}
	Ved omskrivning af definitionen for egenvektor får vi:
	$$(A-\lambda I)x=0$$
	Løsningsrummet $N(A-\lambda I)$ er da egenrummet for matricen $A$. 
	Egenrummet består af alle egenvektorer til en given egenværdi. Da $x\neq 0$
	er løsningen til tidligere nævnt ligning alle de løsninger hvor 
	$(A-\lambda I)=0$.
	
	\section{Det karakteristiske polynomium}
	Egenværdierne for matricen $A$ kan udregnes vha. determinanten for 
	$A-\lambda I$:
	$$det(A-\lambda I)=p(\lambda)=0$$
	Rødderne til det karakteristiske polynomium $p(\lambda)$ vil da være 
	egenværdierne.
	
	\section{Geometrisk og Algebraisk multiplicitet}
	$Geo(\lambda)$ er dimensionen af egenrummet $N(A-\lambda I)$.\\
	$Alg(\lambda)$ er antal gange en given egenværdi optræder.\\
	$$Alg(\lambda)\geq Geo(\lambda)$$
	\\
	Geometrisk og algebraisk multiplicitet er nyttigt ved diagonalisering. En
	matrix $A$ er diagonaliserbar hvis og kun hvis $Geo(\lambda)=Alg(\lambda)$.
	
	\section{Similaritet}
	$A,B\in \mathbb{F}^{n,n}$ $B$ er similær til $A$ hvis der eksisterer
	en ikke-singulær matrix $S$ således at $B=S^-1AS$.
	
	Similaritet betyder medfører at matricerne har samme rank, determinant,
	karakteristisk polynomium, geometrisk multiplicitet.
	
	\section{Theorem 6.1.1}
	Lad $A$ og $B$ være $n \times n$ matricer. Hvis $A$ og $B$
	er similære, så har de to matricer samme karakteristiske
	polynomium og derved også samme egenværdier.
	\\
	\\
	Lad $p_A(x),\, p_B(x)$ være de karakteristiske 
	polynomier for $A$ og $B$. Hvis $B$ er similær til $A$
	så eksisterer der en invertibel matrix $S$ sådan
	at $B=S^{-1}AS$. Derved gælder der:
	\begin{align*}
	p_B(x)&=det(B-\lambda I)\\
	&=det(S^{-1}AS-\lambda I)\\
	&=det(S^{-1})(A-\lambda I)det(S)\\
	&=p_A(x)
	\end{align*}
	Egenværdierne af en matrix er rødderne af det karakteristiske
	polynomium, da de har samme polynomium har de samme egenværdier.
	Egenværdierne 
	
	\section{Diagonaliserbar}
	$A\in \mathbb{F}^{n,n}$, $A$ er diagonaliserbar hvis der findes en ikke-
	singulær matrix $X$ således at:
	$$X^-1AX=D$$
	Hvor $D$ er en diagonalmatrix bestående af $A$'s egenværdier og $X$ 
	diagonaliserer $A$ og har $A$'s egenvektorer som søjlevektorer.
	
	\section{Theorem 6.3.2}
	Først antager vi at $A$ har $n$ lineært uafhængige 
	egenvektorer $(x_1,\dots,x_n)$ med egenværdier 
	$(\lambda_1,\dots,\lambda_n)$\\
	Lad så $X=[x_1,\dots,x_n]$. Det følger så heraf at
	$Ax_j=\lambda_jx_j$ er den j'te søjlevektor for
	$AX$, derved får vi:
	\begin{align*}
	AX&=(Ax_1,\dots,Ax_n)
	&=(\lambda_1x_1,\dots,\lambda_nx_n)
	&=(x_1,\dots,x_n)\begin{bmatrix}
	\lambda_1 & & 0\\
	 & \ddots & \\
	 0 & & \lambda_n
	\end{bmatrix}\\
	&=XD
	\end{align*}
	Siden $X$ har $n$ lineært uafhængige søjlevektorer,
	så følger det at $X$ er invertibel hvorved der gælder:
	$$D=X^{-1}XD=X^{-1}AX$$
	\\
	\\
	Nu antager vi at $A$ er diagonaliserbar. Og vil så vise
	at dette betyder at den har $n$ lineært uafh. egenvek.\\
	Da $A$ er diagonaliserbar, eksisterer der en invertibel
	matrix $X$ sådan at $AX=XD$. Hvis $x_1,\dots,x_n$ er 
	søjlevektorer af $X$, så følger der:
	$$Ax_j=\lambda_j x_j, \quad (\lambda_j = d_{jj}\, (\text{den jj'te indgang i }D))$$
	for ethvert $j$. Da er $\lambda_j$ en egenværdi for $A$
	og søjlerne i $X$, $x_j$, er egenvektorer. Siden søjlevektorerne af $X$
	er lineært uafhængige, så følger det at $A$ har $n$ lineært
	uafh. egenvek.
\end{document}