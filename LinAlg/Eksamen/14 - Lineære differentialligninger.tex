% Dokumentklassen sættes til memoir.
% Manual: http://ctan.org/tex-archive/macros/latex/contrib/memoir/memman.pdf
\documentclass[a4paper,oneside,article]{memoir}

\usepackage{pgf}
\usepackage{tikz}
\usepackage{pgfplots}
\usetikzlibrary{arrows,automata}
\usepackage{verbatim}
 
% Danske udtryk (fx figur og tabel) samt dansk orddeling og fonte med
% danske tegn. Hvis LaTeX brokker sig over æ, ø og å skal du udskifte
% "utf8" med "latin1" eller "applemac". 
\usepackage[utf8]{inputenc}
\usepackage[danish]{babel}
\usepackage[T1]{fontenc}
 
% Matematisk udtryk, fede symboler, theoremer og fancy ting (fx kædebrøker)
\usepackage{amsmath,amssymb}
\usepackage{bm}
\usepackage{amsthm}
%\usepackage{mathtools}
 
% Kodelisting. Husk at læse manualen hvis du vil lave fancy ting.
% Manual: http://mirror.ctan.org/macros/latex/contrib/listings/listings.pdf
\usepackage{listings}
 
% Fancy ting med enheder og datatabeller. Læs manualen til pakken
% Manual: http://www.ctan.org/tex-archive/macros/latex/contrib/siunitx/siunitx.pdf
%\usepackage{siunitx}

% Indsættelse af grafik.
\usepackage{graphicx}
\usepackage{float}
\usepackage{caption}
%\usepackage{subcaption}
 
% Reaktionsskemaer. Læs manualen for at se eksempler.
% Manual: http://www.ctan.org/tex-archive/macros/latex/contrib/mhchem/mhchem.pdf
%\usepackage[version=3]{mhchem}
%\usepackage[noend]{algpseudocode}
%\usepackage{algorithm}

\usepackage{xcolor,colortbl}

\usepackage{listings}

\definecolor{javared}{rgb}{0.6,0,0} % for strings
\definecolor{javagreen}{rgb}{0.25,0.5,0.35} % comments
\definecolor{javapurple}{rgb}{0.5,0,0.35} % keywords
\definecolor{javadocblue}{rgb}{0.25,0.35,0.75} % javadoc

\lstset{language=Java,
basicstyle=\small, %\ttfamily,
keywordstyle=\color{javapurple}\bfseries,
stringstyle=\color{javared},
commentstyle=\color{javagreen},
morecomment=[s][\color{javadocblue}]{/**}{*/},
numbers=left,
numberstyle=\tiny\color{black},
stepnumber=1,
numbersep=10pt,
tabsize=4,
showspaces=false,
showstringspaces=false}

\newcommand{\notimplies}{%
  \mathrel{{\ooalign{\hidewidth$\not\phantom{=}$\hidewidth\cr$\implies$}}}}
  
\newcommand{\inner}[2]{\langle #1,#2 \rangle}

\begin{document}
    \title{Lineær algebra noter - Lineære differentialligninger}
    \author{Lukas Peter Jørgensen, 201206057, DA4
            }
    \maketitle
    
    \tableofcontents
        
    \chapter{Disposition}
    \begin{enumerate}
    \item TBD
    \end{enumerate}
    
	\chapter{Noter}
	
	\section{Lineært differentialligningssystem}
	Der er forskel på et differentialligningssystem 
	og et lineært ligningssystem.\\
	Et lineært differentialligningssystem er et system 
	af $m$ ligninger med $n$ ubekendte, hvor disse kan 
	skrives som:
	\begin{align*}
	y_1'=a_{11}y_1+a_{12}y_2+\dots + a_{1n}y_n\\
	y_2'=a_{21}y_1+a_{22}y_2+\dots + a_{2n}y_n\\
	\dots\\
	y_m'=a_{m1}y_1+a_{m2}y_2+\dots + a_{mn}y_n
	\end{align*}
	$$y_i:I\rightarrow \mathbb{K}^n$$
	Kan også skrives på matrix form som:
	
	\begin{equation}
	Y'=AY
	\end{equation}
	
	\section{Løsningsform}
	For $n=1$ kender vi løsningen som:
	$$y'=ay\implies y(t)=ce^{at}$$
	En generel løsning for $n>1$ er:
	$$Y=\begin{bmatrix}
	x_1e^{\lambda t}\\
	\vdots\\
	x_ne^{\lambda t}
	\end{bmatrix}=e^{\lambda t}x$$
	
	\section{Bevis for løsningsform}
	Hvis man differentiere $e^{\lambda t}x$ får vi:
	$$Y'=\lambda e^{\lambda t}x=\lambda Y$$
	Vi vil nu vise $AY=\lambda Y=Y'$, hvilket vi
	kan gøre hvis vi vælger $\lambda$ til at være en
	egenværdi for $A$ og $x$ den tilhørende egenvektor.
	Så får vi:
	$$AY=Ae^{\lambda t}x=\lambda e^{\lambda t}x = \lambda Y = Y'$$
	$A e^{\lambda t}x$ svarer til at gange a på en
	skaleret egenvektor derfor kan $A$ erstattes
	med $\lambda$.
	
	\section{Lineærkombination er også en løsning}
	Hvis $Y_1$ og $Y_2$ begge er løsninger til $Y'=AY$, så er
	lineærkombinationen af disse også en løsning.
	
	\section{Bevis for lineærkombination}
	Vi skal vise at $Y'=A(\alpha Y_1+ \beta Y_2)$
	\begin{align*}
	(\alpha Y_1 + \beta Y_2)' &= \alpha Y_1'+ \beta Y_2'\\
	&= \alpha A Y_1+\beta AY_2\\
	&=A(\alpha Y_1 + \beta Y_2)
	\end{align*}
	
	\textit{Nu er du fucked.}
\end{document}