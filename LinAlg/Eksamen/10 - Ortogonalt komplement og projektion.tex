% Dokumentklassen sættes til memoir.
% Manual: http://ctan.org/tex-archive/macros/latex/contrib/memoir/memman.pdf
\documentclass[a4paper,oneside,article]{memoir}

\usepackage{pgf}
\usepackage{tikz}
\usepackage{pgfplots}
\usetikzlibrary{arrows,automata}
\usepackage{verbatim}
 
% Danske udtryk (fx figur og tabel) samt dansk orddeling og fonte med
% danske tegn. Hvis LaTeX brokker sig over æ, ø og å skal du udskifte
% "utf8" med "latin1" eller "applemac". 
\usepackage[utf8]{inputenc}
\usepackage[danish]{babel}
\usepackage[T1]{fontenc}
 
% Matematisk udtryk, fede symboler, theoremer og fancy ting (fx kædebrøker)
\usepackage{amsmath,amssymb}
\usepackage{bm}
\usepackage{amsthm}
%\usepackage{mathtools}
 
% Kodelisting. Husk at læse manualen hvis du vil lave fancy ting.
% Manual: http://mirror.ctan.org/macros/latex/contrib/listings/listings.pdf
\usepackage{listings}
 
% Fancy ting med enheder og datatabeller. Læs manualen til pakken
% Manual: http://www.ctan.org/tex-archive/macros/latex/contrib/siunitx/siunitx.pdf
%\usepackage{siunitx}

% Indsættelse af grafik.
\usepackage{graphicx}
\usepackage{float}
\usepackage{caption}
%\usepackage{subcaption}
 
% Reaktionsskemaer. Læs manualen for at se eksempler.
% Manual: http://www.ctan.org/tex-archive/macros/latex/contrib/mhchem/mhchem.pdf
%\usepackage[version=3]{mhchem}
%\usepackage[noend]{algpseudocode}
%\usepackage{algorithm}

\usepackage{xcolor,colortbl}

\usepackage{listings}

\definecolor{javared}{rgb}{0.6,0,0} % for strings
\definecolor{javagreen}{rgb}{0.25,0.5,0.35} % comments
\definecolor{javapurple}{rgb}{0.5,0,0.35} % keywords
\definecolor{javadocblue}{rgb}{0.25,0.35,0.75} % javadoc

\lstset{language=Java,
basicstyle=\small, %\ttfamily,
keywordstyle=\color{javapurple}\bfseries,
stringstyle=\color{javared},
commentstyle=\color{javagreen},
morecomment=[s][\color{javadocblue}]{/**}{*/},
numbers=left,
numberstyle=\tiny\color{black},
stepnumber=1,
numbersep=10pt,
tabsize=4,
showspaces=false,
showstringspaces=false}

\newcommand{\notimplies}{%
  \mathrel{{\ooalign{\hidewidth$\not\phantom{=}$\hidewidth\cr$\implies$}}}}
  
\newcommand{\inner}[2]{\langle #1,#2 \rangle}

\begin{document}
    \title{Lineær algebra noter - Ortogonalt komplement og projektion}
    \author{Lukas Peter Jørgensen, 201206057, DA4
            }
    \maketitle
    
    \tableofcontents
        
    \chapter{Disposition}
    \begin{enumerate}
    \item TBD
    \end{enumerate}
    
	\chapter{Noter}
	
	\section{Ortogonalt komplement}
	Det ortogonale komplement $W^\perp$ til underrummet $W$ af
	$\mathbb{R}^n$ er mængden af alle de vektorer der er ortogonale
	på alle vektorer i $W$.
	$$W^\perp = \{v\in \mathbb{R}^n |v^Tu=0 \forall u \in W \}$$
	Dette medfører: 
	\begin{enumerate}
	\item $W^\perp$ er et underrum af $\mathbb{R}^n$.
	\item $dim(W^\perp)=n-dim(W)$.
	\item $(W^\perp)^\perp$ - hvilket betyder det ortogonale komplement
	af $W^\perp$ er $W$.
	\item Enhver vektor $b$ i $\mathbb{R}^n$ kan blive udtrykt $b=b_w+b_{w^\perp}$
	for $b_w\in W$ og $b_{w^\perp} \in W$ 
	\end{enumerate}
	
	\section{Theorem 5.2.4}
	Hvis $S$ er et underrum af $\mathbb{R}^n$, så er $(S^\perp)^\perp=S$.
	\\
	\\
	Hvis $x \in S$ så er $x$ ortogonal til enhver $y$ i $S^\perp$.
	Derved er $S\subseteq (S^\perp)^\perp$.\\
	\\
	Omvendt, hvis $z\in (S^\perp)^\perp$ kan $z$ skrives som
	$z=u+v,\quad u\in S, v\in S^\perp$ fordi at 
	$\mathbb{R}^n=S\bigoplus S^\perp$. Da $v\in S^\perp$ er 
	ortogonal til både $u$ og $z$. Så følger det at:\\
	($x^Ty=\inner{x}{y}$ for $\mathbb{R}$)
	$$0=v^Tz=v^Tu+v^Tv=v^Tv$$
	Derved må $v=0$. Vi får så at $z=u\in S$ derved 
	er $S=(S^\perp)^\perp$
	
	\section{Korollar 5.5.9}
	$\{u_1,\dots,u_m\}$ er en ortonormal basis for 
	$S\in \mathbb{R}^m$ og $b \in \mathbb{R}^m$. $U=(u_1,\dots,u_m)$
	\\
	\\
	Theorem 5.5.8 siger at projektionen $p$ af $b$ på $S$ er:
	$$p=c_1u_1+\cdots+c_mu_m=Uc$$
	Hvor
	$$c=\begin{bmatrix}
	c_1 \\
	\vdots \\
	c_m
	\end{bmatrix}=\begin{bmatrix}
	u^T_1\\
	\vdots \\
	u^T_m
	\end{bmatrix}=U^Tb$$
	Derved får vi: 
	$$p=UU^Tb$$
	
	\section{Projektionsmatrix}
	Hvis $\{u_1,u_2,\dots,u_k\}$ er en ortonormal basis for $S\in \mathbb{R}^m\\0$
	og $U=(u_1,u_2,\dots,u_k)$ så er projektionen $p$ af $b \in \mathbb{R}^m$
	på $S$ givet ved:
	$$p=UU^T b$$
	Matricen $UU^T$ er en projektionsmatrix til underrummet $S$,
	denne projektionsmatrix er unik.
	
	\section{Bevis for unikhed}
	Hvis $P$ er en projektionsmatrix tilhørende et 
	underrum $S$ af $R^m$, så er projektionen $p$
	af $b$ på $S$ unik.\\
	Hvis $Q$ også er en projektionsmatrix tilhørende $S$
	så er:
	$$Qb=p=Pb$$
	Det følger heraf at:
	$$q_j=Qc_j=Pc_jp_j$$
\end{document}