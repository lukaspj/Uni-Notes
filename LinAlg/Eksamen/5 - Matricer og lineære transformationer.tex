% Dokumentklassen sættes til memoir.
% Manual: http://ctan.org/tex-archive/macros/latex/contrib/memoir/memman.pdf
\documentclass[a4paper,oneside,article]{memoir}

\usepackage{pgf}
\usepackage{tikz}
\usepackage{pgfplots}
\usetikzlibrary{arrows,automata}
\usepackage{verbatim}
 
% Danske udtryk (fx figur og tabel) samt dansk orddeling og fonte med
% danske tegn. Hvis LaTeX brokker sig over æ, ø og å skal du udskifte
% "utf8" med "latin1" eller "applemac". 
\usepackage[utf8]{inputenc}
\usepackage[danish]{babel}
\usepackage[T1]{fontenc}
 
% Matematisk udtryk, fede symboler, theoremer og fancy ting (fx kædebrøker)
\usepackage{amsmath,amssymb}
\usepackage{bm}
\usepackage{amsthm}
%\usepackage{mathtools}
 
% Kodelisting. Husk at læse manualen hvis du vil lave fancy ting.
% Manual: http://mirror.ctan.org/macros/latex/contrib/listings/listings.pdf
\usepackage{listings}
 
% Fancy ting med enheder og datatabeller. Læs manualen til pakken
% Manual: http://www.ctan.org/tex-archive/macros/latex/contrib/siunitx/siunitx.pdf
%\usepackage{siunitx}

% Indsættelse af grafik.
\usepackage{graphicx}
\usepackage{float}
\usepackage{caption}
%\usepackage{subcaption}
 
% Reaktionsskemaer. Læs manualen for at se eksempler.
% Manual: http://www.ctan.org/tex-archive/macros/latex/contrib/mhchem/mhchem.pdf
%\usepackage[version=3]{mhchem}
%\usepackage[noend]{algpseudocode}
%\usepackage{algorithm}

\usepackage{xcolor,colortbl}

\usepackage{listings}

\definecolor{javared}{rgb}{0.6,0,0} % for strings
\definecolor{javagreen}{rgb}{0.25,0.5,0.35} % comments
\definecolor{javapurple}{rgb}{0.5,0,0.35} % keywords
\definecolor{javadocblue}{rgb}{0.25,0.35,0.75} % javadoc

\lstset{language=Java,
basicstyle=\small, %\ttfamily,
keywordstyle=\color{javapurple}\bfseries,
stringstyle=\color{javared},
commentstyle=\color{javagreen},
morecomment=[s][\color{javadocblue}]{/**}{*/},
numbers=left,
numberstyle=\tiny\color{black},
stepnumber=1,
numbersep=10pt,
tabsize=4,
showspaces=false,
showstringspaces=false}

\newcommand{\notimplies}{%
  \mathrel{{\ooalign{\hidewidth$\not\phantom{=}$\hidewidth\cr$\implies$}}}}
  
\newcommand{\inner}[2]{\langle #1,#2 \rangle}

\begin{document}
    \title{Lineær algebra noter - Matricer og lineære transformationer}
    \author{Lukas Peter Jørgensen, 201206057, DA4
            }
    \maketitle
        	
    \tableofcontents
        
    \chapter{Disposition}
    \begin{enumerate}
    	\item TBD
    \end{enumerate}
    
	\chapter{Noter}
	
	\section{Lineær transformation}
	En afbildning $L$ fra et vektorrum $V$ til $W$ kaldes en lineær
	transformation hvis den respekterer lineær struktur, dvs.:
	$$L(\alpha v_1 + \beta v_2) = \alpha L(v_1) + \beta L(v_2)$$
	$\forall v_1,v_2\in V \wedge \alpha,\beta \in \mathbb{F}$\\
    \\
    Lineære transformationer fra $V$ til $W$ skrives som:
    $$L:V\rightarrow W$$
    Af definitionen for lineære transformation følger:
    $$L(0_v)=0_w$$
    $$L(\sum\limits_{i=1}^{n}a_iv_i) = \sum\limits_{i=1}^{n}a_iL(v_i)$$
    $$L(-v)=-L(v)$$
    
    \section{Kernen}
    Kernen $ker(L)$ er alle de vektorer $v$ hvor $L(v)$ giver nulvektoren. 
    Skrevet formelt som:
    
    Lad $L:V\rightarrow W$. Da er kernen af $L$:
    $$ker(L)=\{v\in V|L(v)=0_w \}$$
    
    \section{Billedet/Range}
    $L:V\rightarrow W$ er en lineær tranformation og lad $S$ være et
    underrum af $V$. Billedet af $S$, skrevet $L(S)$ er defineret som:
    $$L(S)=\{w\in W|w=L(v), v\in S \}$$
    Billedet af et komplet vektorrum, $L(V)$, kaldes for range $L$.
    
    \section{Theorem 4.1.1}
    Hvis $L:V\rightarrow W$ er en lineær transformation og $S$ er et underrum af $V$, så gælder der:
    \begin{enumerate}[i]
    \item $ker(L)$ er et underrum af $V$.
    \item $L(s)$ er et underrum af $W$.
    \end{enumerate}
    Det er trivielt at $ker(L)$ ikke er tom, da nulvektoren $0_v$
    er i $ker(L)$.\\
    For at bevise $(i)$ skal $ker(L)$ være lukket under 
    skalarmultiplikation og  addition af vektorer. Hvis $v\in ker(L)$
    og $\alpha$ er en skalar, så gælder der:
    $$L(\alpha v)=\alpha L(v) = \alpha 0_w=0_w$$
    Derfor er $\alpha v \in ker(L)$. Altså er $ker(L)$ lukket 
    under skalarmultiplikation.
    
    $v_1,v_2\in ker(L)$ så gælder der:
    $$L(v_1+v_2)=L(v_1)+L(v_2)=0_w+0_w=0_w$$
    Derfor lukket under vektoraddition.
    \\
    \\
    Beviset for $(ii)$ minder om det foregående. $L(S)$ er ikke-tom
    da $0_w=L(0_v)\in L(S)$. Hvis $w\in L(S)$, så er $w=L(v)$ for
    nogle $v\in S$\\
    For $\alpha in \mathbb{F}$ gælder der:
    $$\alpha w = \alpha L(v) = L(\alpha v)$$
    Siden $\alpha v \in S$ så følger det at $\alpha w \in L(S)$ - 
    lukket under skalar.\\
    Hvis $w_1,w_2 \in L(S)$, så eksisterer $v_1,v_2 \in S$ således
    at $L(v_1)=w_1$ og $L(v_2)=w_2$. Altså gælder der:
    $$w_1+w_2=L(v_1)+L(v_2)=L(v_1+v_2)$$
    Derfor lukket under addition.
    
    \section{Theorem 4.2.1}
    Hvis $L:\mathbb{R}^n\rightarrow \mathbb{R}^m$ så eksisterer $A\in \mathbb{R}^{m,n}$ således at:
    $$L(x)=Ax, \forall x \in \mathbb{R}^n$$
    Faktisk er den j'te søjlevektor for $A$ givet ved:
    $$a_j=L(e_j)\text{ for } j=1,2,\dots,n$$
    \\
    \\
    For $j=1,\dots,n$ defineres:
    $$a_j=L(e_j)$$
    matricen $A$ dannes ved:
    $$A=(a_{ij})=(a_1,a_2,\dots,a_n)$$
    En arbitrær vektor $x\in \mathbb{R}^n$ kan skrives som
    koefficienter ganget med elementær vektorer:
    $$x=x_1e_1+x_2e_2+\dots+x_ne_n$$
    Hvor $e_i$ er:
    $$e_i=\begin{pmatrix}
    0\\
    \vdots\\
    1 \\
    \vdots \\
    0
    \end{pmatrix}\leftarrow\text{ i'te indgang}$$
    Så gælder der:
    \begin{align*}
    	L(x)&=x_1L(e_1)+x_2L(e_2)+\dots+x_nL(e_n)\\
    	&=x_1a_1+x_2a_2+\dots+x_na_n\\
    	&=(a_1,a_2,\dots,a_n)\begin{bmatrix}
    	x_1\\
    	x_2\\
    	\vdots\\
    	x_n
    	\end{bmatrix}\\
    	&=Ax
    \end{align*}
    
\end{document}