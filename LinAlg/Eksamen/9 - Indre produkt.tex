% Dokumentklassen sættes til memoir.
% Manual: http://ctan.org/tex-archive/macros/latex/contrib/memoir/memman.pdf
\documentclass[a4paper,oneside,article]{memoir}

\usepackage{pgf}
\usepackage{tikz}
\usepackage{pgfplots}
\usetikzlibrary{arrows,automata}
\usepackage{verbatim}
 
% Danske udtryk (fx figur og tabel) samt dansk orddeling og fonte med
% danske tegn. Hvis LaTeX brokker sig over æ, ø og å skal du udskifte
% "utf8" med "latin1" eller "applemac". 
\usepackage[utf8]{inputenc}
\usepackage[danish]{babel}
\usepackage[T1]{fontenc}
 
% Matematisk udtryk, fede symboler, theoremer og fancy ting (fx kædebrøker)
\usepackage{amsmath,amssymb}
\usepackage{bm}
\usepackage{amsthm}
%\usepackage{mathtools}
 
% Kodelisting. Husk at læse manualen hvis du vil lave fancy ting.
% Manual: http://mirror.ctan.org/macros/latex/contrib/listings/listings.pdf
\usepackage{listings}
 
% Fancy ting med enheder og datatabeller. Læs manualen til pakken
% Manual: http://www.ctan.org/tex-archive/macros/latex/contrib/siunitx/siunitx.pdf
%\usepackage{siunitx}

% Indsættelse af grafik.
\usepackage{graphicx}
\usepackage{float}
\usepackage{caption}
%\usepackage{subcaption}
 
% Reaktionsskemaer. Læs manualen for at se eksempler.
% Manual: http://www.ctan.org/tex-archive/macros/latex/contrib/mhchem/mhchem.pdf
%\usepackage[version=3]{mhchem}
%\usepackage[noend]{algpseudocode}
%\usepackage{algorithm}

\usepackage{xcolor,colortbl}

\usepackage{listings}

\definecolor{javared}{rgb}{0.6,0,0} % for strings
\definecolor{javagreen}{rgb}{0.25,0.5,0.35} % comments
\definecolor{javapurple}{rgb}{0.5,0,0.35} % keywords
\definecolor{javadocblue}{rgb}{0.25,0.35,0.75} % javadoc

\lstset{language=Java,
basicstyle=\small, %\ttfamily,
keywordstyle=\color{javapurple}\bfseries,
stringstyle=\color{javared},
commentstyle=\color{javagreen},
morecomment=[s][\color{javadocblue}]{/**}{*/},
numbers=left,
numberstyle=\tiny\color{black},
stepnumber=1,
numbersep=10pt,
tabsize=4,
showspaces=false,
showstringspaces=false}

\newcommand{\notimplies}{%
  \mathrel{{\ooalign{\hidewidth$\not\phantom{=}$\hidewidth\cr$\implies$}}}}
  
\newcommand{\inner}[2]{\langle #1,#2 \rangle}

\begin{document}
    \title{Lineær algebra noter - Indre produkt}
    \author{Lukas Peter Jørgensen, 201206057, DA4
            }
    \maketitle
    
    \tableofcontents
        
    \chapter{Disposition}
    \begin{enumerate}
    \item TBD
    \end{enumerate}
    
	\chapter{Noter}
	
	\section{Reelt indre produkt}
	Et indre produkt på vektorrummet $V$ er en operation på $V$
	der tildeler et reelt tal til ethvert par af vektorer. Det
	skrives således: $\inner{x}{y}$
	Der gælder følgende regler:
	\begin{enumerate}
	\item $\inner{x}{x} \geq 0$ med lighed hvis og kun hvis $x=0$.
	\item $\inner{x}{y}=\inner{y}{x}, \forall x,y\in V$ 
	\item $\inner{\alpha x + \beta y}{z}=
	\alpha\inner{x}{z}+\beta\inner{y}{z}, 
	\forall x,y,z \in V \quad \wedge \quad \alpha, \beta \in \mathbb{F}$
	\end{enumerate}
	
	Et vektorrum med et indre produkt kaldes et indre produktrum.
	Som eksempel har $\mathbb{R}^n$ det indre produkt defineret
	som $\inner{x}{y} = y^Tx$.
	
	Det indre produkt for komplekse tal er defineret som:
	$$\inner{u}{v}=\sum\limits_{i=1}^{n}\bar{u_i}v_i$$
	Desuden gælder der for $\mathbb{C}^n$ at: $\inner{u}{v} = 
	\bar{\inner{v}{u}}$. $\mathbb{C}^n$ har det indre produkt
	defineret som $\inner{x}{y}=y^Hx$
	
	\section{Norm og ortogonalitet}
	Længden, eller norm, af $v\in V$ hvor V er et indre produktrum er:
	$$\|v\|=\sqrt{\inner{v}{v}}$$
	To vektorer $v$ og $u$ er ortogonale såfremt $\inner{v}{u}=0$
	
	\section{Theorem 5.4.1}
	\textbf{(Pythagoras)} Hvis $u$ og $v$ er ortogonale 
	($\inner{u}{v}=0$) i et inder produktrum $V$ så gælder der:
	$$\|u+v\|^2=\|u\|^2+\|v\|^2$$
	\\
	\\
	\begin{align*}
	\|u+v\|^2&=\inner{u+v}{u+v}\\
	&= \inner{u}{u} + \inner{u}{v} + \inner{v}{u} + \inner{v}{v}
	&= \|u\|^2+\|v\|^2
	\end{align*}
	Altså gælder Pythagoras' lov for et givent indre produktrum.
	
	\section{Projektion}
	$V\subseteq \mathbb{C}$ er et indreproduktrum.\\
	Hvis $u,v\in V,\, v\neq 0$ så er skalarprojektionen af
	$u$ på $v$:
	$$\alpha = \frac{\inner{u}{v}}{\|v\|}$$
	Vektor projektionen er:
	$$p = \alpha\left(\frac{1}{\|v\|}v\right) = 
	\frac{\inner{u}{v}}{\inner{v}{v}}v$$
	
	\section{Theorem 5.4.2}
	Hvis $u$ og $v$ er tilfældige vektorer i et indre
	produktrum $V$, så gælder der:
	$$|\inner{u}{v}|\leq \|u\|\cdot\|v\|$$
	Lighed iff $u$ og $v$ er lin. afh.
	
	\subsection{Lemma. Egenskaber for projektionsvektorer}
	$u,v\in V$ og $p$ er projektionen $u$ på $v$ så gælder der:
	\begin{enumerate}
	\item $u-p$ og $p$ er ortogonale ($\inner{u-p}{p}=0$).
	\item $u=p$ iff $u=\alpha v$
	\end{enumerate}
	
	\textbf{Bevis for theoremet}\\
	Hvis $v=0$, så er det trivielt at se, at:
	$$|\inner{u}{v}|=0=\|u\|\cdot \|v\|$$
	Hvis $v\neq 0$, lader vi $p$ være en vektor
	projektion af $u$ på $v$.\\
	Siden $p$ er ortogonal til $u-p$ (lemma 1.),
	så følger det af Pythagoras at:
	$$\|p\|^2+\|u-p\|^2=\|u-p+p\|^2=\|u\|^2$$
	Derved får vi:
	$$\frac{\inner{u}{v}^2}{\|v\|^2}=\|p\|^2=\|u\|^2-\|u-p\|^2$$
	Vi kan gange $\|v\|^2$ ind for at få $\inner{u}{v}^2$
	og får så:
	$$\inner{u}{v}^2=\|u\|^2\|v\|^2-\|u-p\|^2\|v\|^2\leq \|u\|^2\|v\|^2$$
	Da vi ved at $\|u-p\|^2\|v\|^2$ er positivt 
	da det er i anden. Vi får så til sidst:
	$$|\inner{u}{v}|\leq\|u\|\cdot\|v\|$$
	Hvor lighed kun er gældenden når $u=p$ fordi
	så bliver $\|u-p\|^2=0$. Det følger så af
	lemmaet at ligheden gælder hvis $v=0$ eller
	$u=\alpha v$ altså hvis de er lineært afhængige.
	
\end{document}