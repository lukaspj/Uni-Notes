% Dokumentklassen sættes til memoir.
% Manual: http://ctan.org/tex-archive/macros/latex/contrib/memoir/memman.pdf
\documentclass[a4paper,oneside,article]{memoir}

\usepackage{pgf}
\usepackage{tikz}
\usepackage{pgfplots}
\usetikzlibrary{arrows,automata}
\usepackage{verbatim}
 
% Danske udtryk (fx figur og tabel) samt dansk orddeling og fonte med
% danske tegn. Hvis LaTeX brokker sig over æ, ø og å skal du udskifte
% "utf8" med "latin1" eller "applemac". 
\usepackage[utf8]{inputenc}
\usepackage[danish]{babel}
\usepackage[T1]{fontenc}
 
% Matematisk udtryk, fede symboler, theoremer og fancy ting (fx kædebrøker)
\usepackage{amsmath,amssymb}
\usepackage{bm}
\usepackage{amsthm}
%\usepackage{mathtools}
 
% Kodelisting. Husk at læse manualen hvis du vil lave fancy ting.
% Manual: http://mirror.ctan.org/macros/latex/contrib/listings/listings.pdf
\usepackage{listings}
 
% Fancy ting med enheder og datatabeller. Læs manualen til pakken
% Manual: http://www.ctan.org/tex-archive/macros/latex/contrib/siunitx/siunitx.pdf
%\usepackage{siunitx}

% Indsættelse af grafik.
\usepackage{graphicx}
\usepackage{float}
\usepackage{caption}
%\usepackage{subcaption}
 
% Reaktionsskemaer. Læs manualen for at se eksempler.
% Manual: http://www.ctan.org/tex-archive/macros/latex/contrib/mhchem/mhchem.pdf
%\usepackage[version=3]{mhchem}
%\usepackage[noend]{algpseudocode}
%\usepackage{algorithm}

\usepackage{xcolor,colortbl}

\usepackage{listings}

\definecolor{javared}{rgb}{0.6,0,0} % for strings
\definecolor{javagreen}{rgb}{0.25,0.5,0.35} % comments
\definecolor{javapurple}{rgb}{0.5,0,0.35} % keywords
\definecolor{javadocblue}{rgb}{0.25,0.35,0.75} % javadoc

\lstset{language=Java,
basicstyle=\small, %\ttfamily,
keywordstyle=\color{javapurple}\bfseries,
stringstyle=\color{javared},
commentstyle=\color{javagreen},
morecomment=[s][\color{javadocblue}]{/**}{*/},
numbers=left,
numberstyle=\tiny\color{black},
stepnumber=1,
numbersep=10pt,
tabsize=4,
showspaces=false,
showstringspaces=false}

\newcommand{\notimplies}{%
  \mathrel{{\ooalign{\hidewidth$\not\phantom{=}$\hidewidth\cr$\implies$}}}}
  
\newcommand{\inner}[2]{\langle #1,#2 \rangle}

\begin{document}
    \title{Lineær algebra noter - Lineære uafhængighed}
    \author{Lukas Peter Jørgensen, 201206057, DA4
            }
    \maketitle
        	
    \tableofcontents
        
    \chapter{Disposition}
    \begin{enumerate}
    	\item TBD
    \end{enumerate}
    
	\chapter{Noter}
	
	\section{Lineær uafhængighed}
	Et sæt $\{v_1,v_2,\dots,v_n\}$ af vektorer er lineært uafhængige såfremt der gælder at linear kombinationen imellem dem kun giver $0$ når alle $c_i$'erne er $0$.
	
	Hvis der eksisterer et $c_i\neq 0$ så vil mindst en af vektorerne kunne skrives som en linear kombination af de andre.
	 
	\section{Spanning set}
	$V=span(v_1,v_2,\dots, v_n)$ hvis: $$\forall x \in V, \exists c_1, c_2, \dots, c_n: x=c_1v_1,c_2v_2,\dots,c_nv_n$$
	Altså $x$ er en lineær kombination af vektorerne i spanning sættet.
	
	\section{Basis}
	Et spanning sæt er en basis hvis disse er lineært uafhængige.
	    
	 \section{Theorem 3.3.1}
	 $$x_1,x_2,\dots,x_n \in \mathbb{R}^n$$
	 $$X=(x_1,x_2,\dots,x_n)$$
	 $$x_1,x_2,\dots,x_n \text{ lin. afh. iff } X \text{ singulær}$$
	 \\
	 $$c_1x_1+c_2x_2+\dots+c_nx_n=0$$
	 $$Xc=0$$
	 $Xc$ har ikke-trivielle løsninger hvis og kun hvis 
	 $X$ er singulær. Hvis $X$ ikke er singulær, er den
	 invertibel. Hvis $X$ er invertibel findes der en 
	 invers matrix $X^{-1}$ som man kan gange på ligningen:
	 $$X^{-1}Xc=0X^{-1}$$
	 $$c=0\implies lin. uafh.$$
	 
	 \section{Theorem 3.4.1}
	 Hvis $V=span(v_1,v_2,\dots,v_n)$ så for ethvert sæt af vektorer 
	 i V $(u_1,u_2,\dots,u_m), m>n$ så er $u_i$'erne indbyrdes
	 lineært afhængige.
	 \\
	 \\
	 Da $V=span(v_1,v_2,\dots,\vDash_n)$ kan $u_i$'erne skrives 
	 som en linearkombination af $v_j$'erne.
	 $$u_i=\sum\limits_{j=1}^{n}a_{ij}v_j, \quad a_{ij}\in \mathbb{F}$$
	 For at finde ud af om $u_i$'erne er uafhængige må der ikke
	 være en ikke-triviel løsning til:
	 $$\sum\limits_{i=1}^{m}c_iu_i=0$$
	 Hvis vi erstatter $u_i$ med en linearkombination af $v_j$'erne 
	 får vi:
	 $$\sum\limits_{i=1}^{m}c_i\sum\limits_{j=1}^{n}a_{ij}v_j=
	 \sum\limits_{j=1}^{n}\sum\limits_{i=1}^{m}(a_{ij}c_i)v_j$$
	 \\
	 \\
	 Hvis vi nu nøjes med at kigge på produktet af $a_{ij}c_i$
	 får vi:
	 $$\sum\limits_{i=1}^{m}a_{ij}c_i=0,\text{ for } j=1,\dots,n$$
	 Her er der flere ubekendte end der er ligninger da $m>n$,
	 det er desuden et homogent system ($b=0$) derfor gælder
	 teorem 1.2.1 der siger at der må være en ikke-triviel
	 løsning.
	 \\
	 \\
	 Vi skal nu vise at løsninger til:
	 $$\sum\limits_{i=1}^{m}a_{ij}c_i=0,\text{ for } j=1,\dots,n$$
	 også er løsninger til:
	 $$\hat{c_1}u_1+\hat{c_2}u_2+\dots+\hat{c_m}u_m=0$$
	 Hvor $\{\hat{c_1},\dots,\hat{c_m}\}, c_i \neq 0$ for flere 
	 $c_i$'er.\\
	 Hvilket løsningerne er, da vi kan indsætte 0:
	 $$\sum\limits_{j=1}^{n}\sum\limits_{i=1}^{m}0v_j=0$$
    
    
\end{document}