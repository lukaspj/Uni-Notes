% Dokumentklassen sættes til memoir.
% Manual: http://ctan.org/tex-archive/macros/latex/contrib/memoir/memman.pdf
\documentclass[a4paper,oneside,article]{memoir}

\usepackage{pgf}
\usepackage{tikz}
\usepackage{pgfplots}
\usetikzlibrary{arrows,automata}
\usepackage{verbatim}
 
% Danske udtryk (fx figur og tabel) samt dansk orddeling og fonte med
% danske tegn. Hvis LaTeX brokker sig over æ, ø og å skal du udskifte
% "utf8" med "latin1" eller "applemac". 
\usepackage[utf8]{inputenc}
\usepackage[danish]{babel}
\usepackage[T1]{fontenc}
 
% Matematisk udtryk, fede symboler, theoremer og fancy ting (fx kædebrøker)
\usepackage{amsmath,amssymb}
\usepackage{bm}
\usepackage{amsthm}
%\usepackage{mathtools}
 
% Kodelisting. Husk at læse manualen hvis du vil lave fancy ting.
% Manual: http://mirror.ctan.org/macros/latex/contrib/listings/listings.pdf
\usepackage{listings}
 
% Fancy ting med enheder og datatabeller. Læs manualen til pakken
% Manual: http://www.ctan.org/tex-archive/macros/latex/contrib/siunitx/siunitx.pdf
%\usepackage{siunitx}

% Indsættelse af grafik.
\usepackage{graphicx}
\usepackage{float}
\usepackage{caption}
%\usepackage{subcaption}
 
% Reaktionsskemaer. Læs manualen for at se eksempler.
% Manual: http://www.ctan.org/tex-archive/macros/latex/contrib/mhchem/mhchem.pdf
%\usepackage[version=3]{mhchem}
%\usepackage[noend]{algpseudocode}
%\usepackage{algorithm}

\usepackage{xcolor,colortbl}

\usepackage{listings}

\definecolor{javared}{rgb}{0.6,0,0} % for strings
\definecolor{javagreen}{rgb}{0.25,0.5,0.35} % comments
\definecolor{javapurple}{rgb}{0.5,0,0.35} % keywords
\definecolor{javadocblue}{rgb}{0.25,0.35,0.75} % javadoc

\lstset{language=Java,
basicstyle=\small, %\ttfamily,
keywordstyle=\color{javapurple}\bfseries,
stringstyle=\color{javared},
commentstyle=\color{javagreen},
morecomment=[s][\color{javadocblue}]{/**}{*/},
numbers=left,
numberstyle=\tiny\color{black},
stepnumber=1,
numbersep=10pt,
tabsize=4,
showspaces=false,
showstringspaces=false}

\newcommand{\notimplies}{%
  \mathrel{{\ooalign{\hidewidth$\not\phantom{=}$\hidewidth\cr$\implies$}}}}
  
\newcommand{\inner}[2]{\langle #1,#2 \rangle}

\begin{document}
    \title{Lineær algebra noter - Ortogonale og ortonormale baser}
    \author{Lukas Peter Jørgensen, 201206057, DA4
            }
    \maketitle
    
    \tableofcontents
        
    \chapter{Disposition}
    \begin{enumerate}
    \item TBD
    \end{enumerate}
    
	\chapter{Noter}
	
	\section{Ortogonalt sæt}
	Et sæt af vektorer $\{v_1,v_2,\dots,v_n\}$ er ortogonale hvis:
	$$\inner{v_j}{v_i}=0 \text{ for } i\neq j$$
	Dette sæt er en basis hvis det opfylder definitionen for en
	basis.
	
	\section{Basis}
	Sættet $\{v_1,v_2,\dots,v_n\}$ er en basis for $V$ hvis 
	vektorerne i sættet indbyrdes er lineært uafhængige og spanner
	$V$.
	
	\section{Ortonormalt sæt}
	Et sæt af vektorer $\{v_1,v_2,\dots,v_n\}$ er ortonormalt hvis:
	\begin{itemize}
	\item Sættet er ortogonalt
	\item Sættet består af enhedsvektorer (er normeret).
	\end{itemize}
	$$\inner{v_i}{v_j} = \delta_{ij} \left\{ \begin{array}{ll}
		         1 & \mbox{for i=j},\\
		        0 & \mbox{ellers}\end{array} \right.$$
		        
	\section{Theorem 5.5.1}
	Hvis $\{v_1,\dots,v_n\}$ er et ortogonalt set af ikke-nul
	vektorer i et indre produktrum $V$, så er $v_1,\dots,v_n$ 
	lineært uafhængige.
	\\
	\\
	$v_1,\dots,v_n$ er ortogonale og ikke-nul vektorer.
	$$c_1v_1+\cdots+c_nv_n=0$$
	For $1\leq j \leq n$ får man ved at tage det indreprodukt
	af $v_j$ på begge sider af udregningen:
	$$c_j\inner{v_j}{v_1}+\cdots+c_n\inner{v_j}{v_n}=0$$
	$$c_j\inner{v_j}{v_j}=0$$
	Og derved må alle skalarer være 0 - altså er $v_i$'erne
	uafhængige.
	
	\section{Theorem 5.5.2}
	Lad $\{u_1,\dots,u_n\}$ være en ortonormal basis for et 
	indre produktrum $V$. Hvis der så gælder:
	$$v=\sum\limits_{i=1}^{n}c_iu_i$$
	Så er $c_i=\inner{v}{u_i}$.
	\\
	\\
	Vi udregner $\inner{v}{u_i}$:
	$$\inner{v}{u_i}=\inner{\sum\limits_{j=1}^{n}c_ju_j}{u_i}
	=\sum\limits_{j=1}^{n}(c_j\inner{u_j}{u_i})
	=\sum\limits_{j=1}^{n}c_j\delta_{ji}=c_i$$
	
\end{document}