% Dokumentklassen sættes til memoir.
% Manual: http://ctan.org/tex-archive/macros/latex/contrib/memoir/memman.pdf
\documentclass[a4paper,oneside,article]{memoir}

\usepackage{pgf}
\usepackage{tikz}
\usepackage{pgfplots}
\usetikzlibrary{arrows,automata}
\usepackage{verbatim}
 
% Danske udtryk (fx figur og tabel) samt dansk orddeling og fonte med
% danske tegn. Hvis LaTeX brokker sig over æ, ø og å skal du udskifte
% "utf8" med "latin1" eller "applemac". 
\usepackage[utf8]{inputenc}
\usepackage[danish]{babel}
\usepackage[T1]{fontenc}
 
% Matematisk udtryk, fede symboler, theoremer og fancy ting (fx kædebrøker)
\usepackage{amsmath,amssymb}
\usepackage{bm}
\usepackage{amsthm}
%\usepackage{mathtools}
 
% Kodelisting. Husk at læse manualen hvis du vil lave fancy ting.
% Manual: http://mirror.ctan.org/macros/latex/contrib/listings/listings.pdf
\usepackage{listings}
 
% Fancy ting med enheder og datatabeller. Læs manualen til pakken
% Manual: http://www.ctan.org/tex-archive/macros/latex/contrib/siunitx/siunitx.pdf
%\usepackage{siunitx}

% Indsættelse af grafik.
\usepackage{graphicx}
\usepackage{float}
\usepackage{caption}
%\usepackage{subcaption}
 
% Reaktionsskemaer. Læs manualen for at se eksempler.
% Manual: http://www.ctan.org/tex-archive/macros/latex/contrib/mhchem/mhchem.pdf
%\usepackage[version=3]{mhchem}
%\usepackage[noend]{algpseudocode}
%\usepackage{algorithm}

\usepackage{xcolor,colortbl}

\usepackage{listings}

\definecolor{javared}{rgb}{0.6,0,0} % for strings
\definecolor{javagreen}{rgb}{0.25,0.5,0.35} % comments
\definecolor{javapurple}{rgb}{0.5,0,0.35} % keywords
\definecolor{javadocblue}{rgb}{0.25,0.35,0.75} % javadoc

\lstset{language=Java,
basicstyle=\small, %\ttfamily,
keywordstyle=\color{javapurple}\bfseries,
stringstyle=\color{javared},
commentstyle=\color{javagreen},
morecomment=[s][\color{javadocblue}]{/**}{*/},
numbers=left,
numberstyle=\tiny\color{black},
stepnumber=1,
numbersep=10pt,
tabsize=4,
showspaces=false,
showstringspaces=false}

\newcommand{\notimplies}{%
  \mathrel{{\ooalign{\hidewidth$\not\phantom{=}$\hidewidth\cr$\implies$}}}}
  
\newcommand{\inner}[2]{\langle #1,#2 \rangle}

\begin{document}
    \title{Lineær algebra noter - Unitær diagonalisering}
    \author{Lukas Peter Jørgensen, 201206057, DA4
            }
    \maketitle
    
    \tableofcontents
        
    \chapter{Disposition}
    \begin{enumerate}
    \item TBD
    \end{enumerate}
    
	\chapter{Noter}
	
	\section{Hermitiske matricer}
	$M=(m_{ij})\in \mathbb{C}^{m,n}$. $M$ er hermitisk såfremt matricen
	konjungeret og transponeret er lig sig selv. Dette skrives som $M=M^H$.
	En hermitisk matrix er det samme som en symmetrisk matrix for det
	reelle rum, blot for komplekse tal.
	
	Desuden gælder der at egenværdierne af en hermitisk matrix alle
	er reelle.
		
	\section{Unitære matricer}
	En matrix $U\in \mathbb{C}^n$ er en unitær matrix hvis dens søjlevektorer
	udgører et ortonormalt sæt i $\mathbb{C}^n$.
	
	En matrix er unitær hvis og kun hvis der gælder at $U^HU=I$.
		
	\section{Schurs theorem}
	For enhver $n \times n$ matrix $A$ eksisterer der
	en unitær $n \times n$ matrix $U$, således at 
	$U^HAU$ er øvre triangulær.
	\\
	\\
	Vi beviser Schurs vha. induktion i $n$.
	\\
	\\
	\underline{$n=1$}\\
	At det gælder for $n=1$ er trivielt da en $1 \times 1$
	matrix pr. definition allerede er triangulær, og det
	derfor er let at finde en unitær matrix for dette
	tilfælde.
	\\
	\\
	\textit{\underline{Induktionshypotese:}}\\
	Som induktionshypotese antager vi at påstanden gælder
	for $k \times k$ matricer.
	\\
	\\
	\underline{$n=k+1$}\\
	Vi definerer $A$ som en $k+1 \times k+1$ matrix, hvor
	$\lambda_1$ er en egenværdi og $w_1$ den tilhørende
	egenvektor.\\
	Vha. Gram-Schmidt konstrueres en ortonormal basis for 
	$\mathbb{C}^{k+1}$ bestående af vektorerne
	$\{w_1,\dots,w_{k+1}\}$.\\
	Hvis $W$ så er en matrix med dette sæt som søjlevektorer,
	så er $W$ unitær.\\
	Den første søjle i $W^HAW$ er så:
	$$W^HAw_1=\lambda W^Hw_1=\lambda_1e_1$$
	Derved har matrixen $W^HAW$ formen:
	$$W^HAW=\begin{bmatrix}[c|ccc]
	\lambda_1 & x_2 & \cdots & x_{k+1} \\ \hline
	0 & & & \\
	\vdots & & M & \\
	0 & & &
	\end{bmatrix}$$
	Hvor $M$ er en $k \times k$ matrix. Vha. induktionshypotesen
	ved vi der findes en unitær $k \times k$ matrix $V_1$ 
	således at $V_1^HMV_1=T_1$ er øvre triangulær.\\
	Vi kan så danne $V$:
	$$V=\begin{bmatrix}[c|ccc]
		1 & 0 & \cdots & 0 \\ \hline
		0 & & & \\
		\vdots & & V_1 & \\
		0 & & &
		\end{bmatrix}$$
 Som så vil være unitær g:
 $$V^HW^HAWV=\begin{bmatrix}[c|ccc]
    	\lambda_1 & x_2 & \cdots & x_{k+1} \\ \hline
    	0 & & & \\
    	\vdots & & V_1^HMV_1 & \\
    	0 & & &
    	\end{bmatrix}=
    	\begin{bmatrix}[c|ccc]
   		\lambda_1 & x_2 & \cdots & x_{k+1} \\ \hline
   		0 & & & \\
   		\vdots & & T_1 & \\
   		0 & & &
   		\end{bmatrix}
   		=T$$
   	$T$ er så øvre triangulær. Vi lader nu $U=WV$. Da er 
   	$U$ så unitær siden:
   	$$U^HU=(WV)^HWV=V^HW^HWV=I$$
   	Og $U^HAU=T$.
   	
   	\textbf{Note: Schur decomposition}
   	Faktoriseringen $A=UTU^H$ bliver ofte kaldt 
   	dekompositionen af $A$.
   	
   	\section{Spektralsætningen}
   	Hvis $A$ er hermitisk så eksisterer der en unitær 
   	matrix $U$ der diagonaliserer $A$.
   	\\
   	\\
   	Vha. Schurs theorem ved vi at der eksisterer en
   	unitær matrix $U$ således at $U^HAU=T$, hvor $T$
   	er øvre triangulær. Det følger derfor:
   	$$T^H=(U^HAU)^H=U^HA^HU=U^HAU=T$$
   	(Husk at $(ABC)^H=C^HB^HA^H$)\\
   	Altså er $T$ hermitisk og må derved også være
   	diagonal.
   	$$T=\begin{bmatrix}
   	t_{11} & \cdots & t_{1n} \\
   	0 & \ddots & \vdots \\
   	0 & 0 & t_{nn}
   	\end{bmatrix},
   	T^H=\begin{bmatrix}
	   	\bar{t_{11}} & 0 & 0 \\
	   	\vdots & \ddots & 0 \\
	   	\bar{t_{1n}} & \cdots & \bar{t_{nn}}
	   	\end{bmatrix}$$
	
\end{document}