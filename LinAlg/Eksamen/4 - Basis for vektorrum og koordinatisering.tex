% Dokumentklassen sættes til memoir.
% Manual: http://ctan.org/tex-archive/macros/latex/contrib/memoir/memman.pdf
\documentclass[a4paper,oneside,article]{memoir}

\usepackage{pgf}
\usepackage{tikz}
\usepackage{pgfplots}
\usetikzlibrary{arrows,automata}
\usepackage{verbatim}
 
% Danske udtryk (fx figur og tabel) samt dansk orddeling og fonte med
% danske tegn. Hvis LaTeX brokker sig over æ, ø og å skal du udskifte
% "utf8" med "latin1" eller "applemac". 
\usepackage[utf8]{inputenc}
\usepackage[danish]{babel}
\usepackage[T1]{fontenc}
 
% Matematisk udtryk, fede symboler, theoremer og fancy ting (fx kædebrøker)
\usepackage{amsmath,amssymb}
\usepackage{bm}
\usepackage{amsthm}
%\usepackage{mathtools}
 
% Kodelisting. Husk at læse manualen hvis du vil lave fancy ting.
% Manual: http://mirror.ctan.org/macros/latex/contrib/listings/listings.pdf
\usepackage{listings}
 
% Fancy ting med enheder og datatabeller. Læs manualen til pakken
% Manual: http://www.ctan.org/tex-archive/macros/latex/contrib/siunitx/siunitx.pdf
%\usepackage{siunitx}

% Indsættelse af grafik.
\usepackage{graphicx}
\usepackage{float}
\usepackage{caption}
%\usepackage{subcaption}
 
% Reaktionsskemaer. Læs manualen for at se eksempler.
% Manual: http://www.ctan.org/tex-archive/macros/latex/contrib/mhchem/mhchem.pdf
%\usepackage[version=3]{mhchem}
%\usepackage[noend]{algpseudocode}
%\usepackage{algorithm}

\usepackage{xcolor,colortbl}

\usepackage{listings}

\definecolor{javared}{rgb}{0.6,0,0} % for strings
\definecolor{javagreen}{rgb}{0.25,0.5,0.35} % comments
\definecolor{javapurple}{rgb}{0.5,0,0.35} % keywords
\definecolor{javadocblue}{rgb}{0.25,0.35,0.75} % javadoc

\lstset{language=Java,
basicstyle=\small, %\ttfamily,
keywordstyle=\color{javapurple}\bfseries,
stringstyle=\color{javared},
commentstyle=\color{javagreen},
morecomment=[s][\color{javadocblue}]{/**}{*/},
numbers=left,
numberstyle=\tiny\color{black},
stepnumber=1,
numbersep=10pt,
tabsize=4,
showspaces=false,
showstringspaces=false}

\newcommand{\notimplies}{%
  \mathrel{{\ooalign{\hidewidth$\not\phantom{=}$\hidewidth\cr$\implies$}}}}
  
\newcommand{\inner}[2]{\langle #1,#2 \rangle}

\begin{document}
    \title{Lineær algebra noter - Basis og koordinatisering}
    \author{Lukas Peter Jørgensen, 201206057, DA4
            }
    \maketitle
        	
    \tableofcontents
        
    \chapter{Disposition}
    \begin{enumerate}
    	\item TBD
    \end{enumerate}
    
	\chapter{Noter}
	
	\section{Ordnet basis}
	Sættet $\{b_1,b_2,\dots,b_n\}$ er en basis for $V$ hvis 
	de er lineært uafhængige.
	
	Normalt er rækkefølgen i en basis irrelevant, men i visse
	tilfælde (deriblandt for koordinatisering) er det nødvendigt
	at have dem ordnet: $[b_1,b_2,\dots,b_n]$.
	
	 \section{Theorem 3.4.1}
	 Hvis $V=span(v_1,v_2,\dots,v_n)$ så for ethvert sæt af vektorer 
	 i V $(u_1,u_2,\dots,u_m), m>n$ så er $u_i$'erne indbyrdes
	 lineært afhængige.
	 \\
	 \\
	 Da $V=span(v_1,v_2,\dots,\vDash_n)$ kan $u_i$'erne skrives 
	 som en linearkombination af $v_j$'erne.
	 $$u_i=\sum\limits_{j=1}^{n}a_{ij}v_j, \quad a_{ij}\in \mathbb{F}$$
	 For at finde ud af om $u_i$'erne er uafhængige må der ikke
	 være en ikke-triviel løsning til:
	 $$\sum\limits_{i=1}^{m}c_iu_i=0$$
	 Hvis vi erstatter $u_i$ med en linearkombination af $v_j$'erne 
	 får vi:
	 $$\sum\limits_{i=1}^{m}c_i\sum\limits_{j=1}^{n}a_{ij}v_j=
	 \sum\limits_{j=1}^{n}\sum\limits_{i=1}^{m}(a_{ij}c_i)v_j$$
	 \\
	 \\
	 Hvis vi nu nøjes med at kigge på produktet af $a_{ij}c_i$
	 får vi:
	 $$\sum\limits_{i=1}^{m}a_{ij}c_i=0,\text{ for } j=1,\dots,n$$
	 Her er der flere ubekendte end der er ligninger da $m>n$,
	 det er desuden et homogent system ($b=0$) derfor gælder
	 teorem 1.2.1 der siger at der må være en ikke-triviel
	 løsning.
	 \\
	 \\
	 Vi skal nu vise at løsninger til:
	 $$\sum\limits_{i=1}^{m}a_{ij}c_i=0,\text{ for } j=1,\dots,n$$
	 også er løsninger til:
	 $$\hat{c_1}u_1+\hat{c_2}u_2+\dots+\hat{c_m}u_m=0$$
	 Hvor $\{\hat{c_1},\dots,\hat{c_m}\}, c_i \neq 0$ for flere 
	 $c_i$'er.\\
	 Hvilket løsningerne er, da vi kan indsætte 0:
	 $$\sum\limits_{j=1}^{n}\sum\limits_{i=1}^{m}0v_j=0$$
	
	\section{Koordinatvektor}
	$V$ er et vektorrum og $E=[v_1,v_2,\dots,v_n]$ er en ordnet
	basis for $V$. Hvis $v$ er et element af $V$, så kan $v$ 
	skrives på formen:
	$$v=c_1v_1+c_2v_2+\dots + c_nv_n$$
	hvor $c_1,c_2,\dots c_n$ er skalarer. Vi kan nu associere
	enhver vektor $v$ med en unik vektor $c=(c_1,c_2,\dots c_n)^T$ 
	i $\mathbb{R}^n$. Denne vektor $c$ kaldes koordinatvektoren 
	for $v$ ift. den ordnede basis $E$ og skrives:
	$$[v]_E$$
	$c_i$'erne kaldes koordinaterne for $v$ relativt til $E$.
	
	\section{Lemma 3.3.2}
	Koordinatisering bevarer lineære strukturer:
	\begin{enumerate}
	\item $[v+w]_E=[v]_E+[W]_E$
	\item $[rv]_E=r[v]_E$
	\end{enumerate}
	1:\\
	Lad $v,w\in V$, $v$ og $w$ kan skrives som:
	$$v=c_1v_1+\cdots+c_nv_n,\, w=d_1+\cdots+d_nv_n$$
	Da er $v+w=(c_1+d_1)v_1+\cdots+(c_nd_n)v_n$ og så gælder der:
	$$[v+w]_E=\begin{bmatrix}
	c_1+d_1\\
	\vdots\\
	c_nd_n
	\end{bmatrix}=\begin{bmatrix}
	c_1\\
	\vdots\\
	c_n
	\end{bmatrix}+\begin{bmatrix}
	d_1\\
	\vdots\\
	d_n
	\end{bmatrix}
	=[v]_E+[w]_E$$
	2:\\
	Lad $v\in V$, $v$ kan da skrives som:
	$$v=c_1v_1+\cdots+c_nv_n$$
	Da er $rv=rc_1v_1+\cdots+rc_nv_n$ og
	$$[rv]_E=\begin{bmatrix}
	rc_1\\
	\vdots\\
	rc_n
	\end{bmatrix}=r\begin{bmatrix}
	c_1\\
	\vdots\\
	c_n
	\end{bmatrix}
	=r[v]_E$$
	
	?? ----
	
	Hvis et vektorrum $V$ har $dim(n)$ så kan $V$ koordinatiseres
	så den "efterligner" $\mathbb{R}^n$. Vi kan dermed arbejde med
	koordinatiseringen med samme værktøjer som ved rummet 
	$\mathbb{R}^n$ 
	
	---- ??
    
\end{document}