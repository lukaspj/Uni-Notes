% Dokumentklassen sættes til memoir.
% Manual: http://ctan.org/tex-archive/macros/latex/contrib/memoir/memman.pdf
\documentclass[a4paper,oneside,article]{memoir}

\usepackage{pgf}
\usepackage{tikz}
\usepackage{pgfplots}
\usetikzlibrary{arrows,automata}
\usepackage{verbatim}
 
% Danske udtryk (fx figur og tabel) samt dansk orddeling og fonte med
% danske tegn. Hvis LaTeX brokker sig over æ, ø og å skal du udskifte
% "utf8" med "latin1" eller "applemac". 
\usepackage[utf8]{inputenc}
\usepackage[danish]{babel}
\usepackage[T1]{fontenc}
 
% Matematisk udtryk, fede symboler, theoremer og fancy ting (fx kædebrøker)
\usepackage{amsmath,amssymb}
\usepackage{bm}
\usepackage{amsthm}
%\usepackage{mathtools}
 
% Kodelisting. Husk at læse manualen hvis du vil lave fancy ting.
% Manual: http://mirror.ctan.org/macros/latex/contrib/listings/listings.pdf
\usepackage{listings}
 
% Fancy ting med enheder og datatabeller. Læs manualen til pakken
% Manual: http://www.ctan.org/tex-archive/macros/latex/contrib/siunitx/siunitx.pdf
%\usepackage{siunitx}

% Indsættelse af grafik.
\usepackage{graphicx}
\usepackage{float}
\usepackage{caption}
%\usepackage{subcaption}
 
% Reaktionsskemaer. Læs manualen for at se eksempler.
% Manual: http://www.ctan.org/tex-archive/macros/latex/contrib/mhchem/mhchem.pdf
%\usepackage[version=3]{mhchem}
%\usepackage[noend]{algpseudocode}
%\usepackage{algorithm}

\usepackage{xcolor,colortbl}

\usepackage{listings}

\definecolor{javared}{rgb}{0.6,0,0} % for strings
\definecolor{javagreen}{rgb}{0.25,0.5,0.35} % comments
\definecolor{javapurple}{rgb}{0.5,0,0.35} % keywords
\definecolor{javadocblue}{rgb}{0.25,0.35,0.75} % javadoc

\lstset{language=Java,
basicstyle=\small, %\ttfamily,
keywordstyle=\color{javapurple}\bfseries,
stringstyle=\color{javared},
commentstyle=\color{javagreen},
morecomment=[s][\color{javadocblue}]{/**}{*/},
numbers=left,
numberstyle=\tiny\color{black},
stepnumber=1,
numbersep=10pt,
tabsize=4,
showspaces=false,
showstringspaces=false}

\newcommand{\notimplies}{%
  \mathrel{{\ooalign{\hidewidth$\not\phantom{=}$\hidewidth\cr$\implies$}}}}
  
\newcommand{\inner}[2]{\langle #1,#2 \rangle}

\begin{document}
    \title{Lineær algebra noter - Determinanter}
    \author{Lukas Peter Jørgensen, 201206057, DA4
            }
    \maketitle
    
    \tableofcontents
        
    \chapter{Disposition}
    \begin{enumerate}
    	\item TBD
    \end{enumerate}
    
	\chapter{Noter}
	
	\section{Determinant}
	Determinanten for en $n \times n$ matrix $A$
	er:
	$$det(A)=a_{11}A_{11}+a_{12}A_{12}+ \dots + a_{1n}A_{1n}$$
	Hvor $A_{ij}$ for $1\leq i, j\leq n$ siges at være den $(i,k)$'te
	cofaktor af $A$ givet som:
	$$A_{ij}=(-1)^{i+j}det(M(A)_{ij})$$
	\textbf{OBS: hvis $n=1$ så er determinanten blot $a_{11}$}\\
	En matrix $A$ er invertibel såfremt $det(A)\neq 0$, dette hedder også at $A$ ikke er singulær.
	
	\section{Egenskaber ved determinanter}
	\begin{itemize}
	\item For enhver $n\times n$ matrix $A$ har vi $det(A)=det(A^T)$
	\item Hvis en matrix er triangulær, kan determinanten findes ved 
	blot at finde produktet af diagonalelementerne.
	\item Hvor to rækker el. søjler af en $n \times n$ matrix er ens,
	så er $det(A)=0$.
	\item Hvis to rækker af en $n \times n$ matrix byttes om, så er
	$det(A')=-det(A)$.
	\item Hvis en række ganges med en skalar $r$ bliver determinanten
	$r\cdot det(A)$
	\item Hvis et multiplum af en række bliver adderet til en anden 
	række, så forbliver determinanten den samme.
	\end{itemize}
	De sidste 3 = ERO'er.
	
	\section{Theorem 2.1.2}
	Hvis $A$ er en $n\times n$ matrix så er $det(A^T)=det(A)$.
	\\
	\\
	Vha. induktion.\\
	Basistilfældet er nemt da vi har en $1 \times 1$ 
	matrix som er symmetrisk hvorved $A^T=A \implies 
	det(A^T)=det(A)$.\\
	I induktionshypotesen antager vi nu at det gælder 
	for $k \times k$, så  skal vi i induktionsskridtet 
	se om det holder for $k+1 \times k+1$.\\
	Vi starter med at lave cofaktor-ekspansion på første
	række af den nye $A$:
	$$det(A)=a_{1,1}det(M_{1,1})-a_{1,2}det(M_{1,2})+\dots 
	\pm a_{1,k+1}det(M_{1,k+1}) $$
	$M_{ij}$'erne må være $k \times k$ matricer fordi de
	er er $A$ med en række og en søjle fjernet. ($k+1-1$).
	Så følger det af induktionshypotesen at (da det gælder
	for $k \times k$ matricer):
	$$det(A)=a_{1,1}det(M_{1,1}^T)-a_{1,2}det(M_{1,2}^T)+\dots 
		\pm a_{1,k+1}det(M_{1,k+1}^T) $$
	Nu er cofaktor-ekspansion af første række af $A$ nu 
	blot lig med cofaktor-ekspansion af første søjle af
	$A^T$. Derved må der gælde:
	$$det(A^T)=det(A)$$
	Det er første søjle fordi at man først tager minoren
	og derefter transponerer.
	
	\section{Theorem 2.2.2}
	En $n \times n$ matrix $A$ er singulær hvis og kun hvis:
	$$det(A)=0$$
	\\
	\\
	$A$ kan blive reduceret til REF vha. et endeligt antal
	ERO'er:
	$$U=E_kE_{k-1}\dots E_1A$$
	hvor $U$ er i REF og $E_i$'erne er elementærmatricer. Det
	ses så:
	\begin{align*}
	det(U)&=det(E_kE_{k-1}\dots E_1A)\\
	&=det(E_k)det(E_{k-1})\dots det(E_1)det(A)
	\end{align*}
	Siden determinanterne af $E_i$'erne alle er ikke-nul, så
	følger det at $det(A)=0$ hvis og kun hvis $det(U)=0$.
	Hvis $A$ ikke er singulær, så er $U$ triangulær med $1$'er
	ned langs diagonalen og derved er $det(U)=1$.\\
	Altså er $A$ singulær hvis og kun hvis $det(A)=0$.
	
	\subsection*{Uddyb elementær matricer.. Hvad kaldes $e_i$?}
	
	\section{Note. Adjungerede matrix}
	\textbf{Note:} $For 1\leq i, j\leq n$ er den $(i,j)$'te
	kofaktor $A_{ij}$ af $A$ givet som:
	$$A_{ij}=(-1)^{i+j}det(M(A)_{ij})$$
	
	Den adjungerede til $A \in \mathbb{F}^{n,n}$ er en 
	matrix hvor hver indgang $a_{ij}$ er erstattet med 
	dets kofaktor $A_{ij}$ og matricen er transponeret.
	
	Det følger at:
	$$A(adj(A))=det(A)I$$
	Og hvis A ikke er singulær kan det skrives om til:
	$$I=A(\frac{1}{det(A)}adj(A)) \Leftrightarrow A^{-1}
	=\frac{1}{det(A)}adj(A)$$
    
    \section{Theorem 2.3.1}
    \textbf{(Cramers regel)} Lad $A$ være en $n \times n$ 
    invertibel matrix og lad $b\in \mathbb{R}^n$. Lad
    $A_i$ være matrixen der fås ved at erstatte den $i$'te
    søjle i $A$ med $b$. Den entydige løsning til systemet
    $Ax=b$ er givet ved;
    $$x_i=\frac{det(A_i)}{det(A)},\, i=1,\dots,n$$
    \\
    \\
    $A^{-b}$ er den entydige løsning til $Ax=b$. Så vi har
    $$x=A^{-1}b=\frac{1}{det(A)}adj(A)b$$
    Og for $i=1,\dots,n$
    \begin{align*}
    x&=A^{-1}b=\frac{1}{det(A)}(\text{i'te række i }adj(A))b\\
    &=\frac{1}{det(A)}(A_{1i}b_1+\dots+A_{ni}b_n)\\
    &=\frac{det(A_i)}{det(A)}
    \end{align*}
    $det(A_i)=(A_{1i}b_1+\dots+A_{ni}b_n)$ fordi at $A_i$
    er $A$ med den $i$'te søjle byttet ud med $b$ og
    $adj(A)_{ji}=A_{ji}=(-1)^{i+j}det(M(A)_{ij})$
    
    \section{Eks. Cramers regel $2 \times 2$ matrix}
    Vi antage vi har følgende lineære ligningssystem:
    $$ax+by=e\,cx+dy=f$$
    Som kan skrives som:
    $$\begin{bmatrix}
    a & b\\
    c & d
    \end{bmatrix}\begin{bmatrix}
    x \\ y
    \end{bmatrix}=\begin{bmatrix}
    e \\ f
    \end{bmatrix}$$
    Så kan $x$ og $y$ findes vha. Cramers regel:
    $$x=\frac{\begin{vmatrix}
    e & b \\
    f & d
    \end{vmatrix}}{
    \begin{vmatrix}
    a & b \\
    c & d
    \end{vmatrix}}= \frac{ed-bf}{ad-bc}$$
    Og..
    $$x=\frac{\begin{vmatrix}
    a & e \\
    c & f
    \end{vmatrix}}{
    \begin{vmatrix}
    a & b \\
    c & d
    \end{vmatrix}}= \frac{af-ec}{ad-bc}$$
\end{document}